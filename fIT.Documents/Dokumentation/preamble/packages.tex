
%%% Doc: ftp://tug.ctan.org/pub/tex-archive/macros/latex/required/babel/babel.pdf
% Languagesetting
\usepackage{babel}	% Sprache

\usepackage{textpos} 


\usepackage{fixltx2e}	% Verbessert einige Kernkompetenzen von LaTeX2e
\usepackage{ellipsis}	% Korrigiert den Wei�raum um Auslassungspunkte

\usepackage{placeins} 

\usepackage{ifpdf}
\ifpdf
\pdfinfo {
	/Author (\theauthor)
	/Title (\thetitle)
	/Subject ()
	/Keywords ()
%	/CreationDate (D:YYYYMMTTHHMMSS)
}
\fi



%%% Doc: www.cs.brown.edu/system/software/latex/doc/calc.pdf
% Calculation with LaTeX
\usepackage{calc}

%%% Doc: ftp://tug.ctan.org/pub/tex-archive/macros/latex/contrib/xcolor/xcolor.pdf
% Farben
% Incompatible: Do not load when using pstricks !
\usepackage[
	table % Load for using rowcolors command in tables
]{xcolor}

\usepackage{tikz}
\usetikzlibrary{% 
   arrows,% 
   calc,% 
   fit,% 
   patterns,% 
   plotmarks,% 
   shapes.geometric,% 
   shapes.misc,% 
   shapes.symbols,% 
   shapes.arrows,% 
   shapes.callouts,% 
   shapes.multipart,% 
   shapes.gates.logic.US,% 
   shapes.gates.logic.IEC,% 
   er,% 
   automata,% 
   backgrounds,% 
   chains,% 
   topaths,% 
   trees,% 
   petri,% 
   mindmap,% 
   matrix,% 
   calendar,% 
   folding,% 
   fadings,% 
   through,% 
   positioning,% 
   scopes,% 
   decorations.fractals,% 
   decorations.shapes,% 
   decorations.text,% 
   decorations.pathmorphing,% 
   decorations.pathreplacing,% 
   decorations.footprints,% 
   decorations.markings,% 
   shadows} 
   
%%% Doc: ftp://tug.ctan.org/pub/tex-archive/macros/latex/required/graphics/grfguide.pdf
% Bilder
\usepackage[%
	%final,
	%draft % do not include images (faster)
]{graphicx}


% bessere Abstaende innerhalb der Tabelle (Layout))
% -------------------------------------------------
%%% Doc: ftp://tug.ctan.org/pub/tex-archive/macros/latex/contrib/booktabs/booktabs.pdf
\usepackage{booktabs}


%%% Doc: ftp://tug.ctan.org/pub/tex-archive/macros/latex/contrib/enumitem/enumitem.pdf
% Better than 'paralist' and 'enumerate' because it uses a keyvalue interface!
% Do not load together with enumerate.
%\usepackage{enumitem}

\usepackage{paralist}


%%% Doc: http://www.ctan.org/tex-archive/macros/latex/contrib/acronym/acronym.pdf
% Usage:
%        Definition: \acro{ acronym }[ short name ]{ full name }
%        Nutzung im Text: \ac{acronym}
 \usepackage[
 	footnote,	% Full names appear in the footnote
 	%smaller,		% Print acronym in smaller fontsize
 	%printonlyused %
 ]{acronym}
%\chapter*{Abkürzungsverzeichnis}
\addcontentsline{toc}{chapter}{Abkürzungsverzeichnis}  
\begin{acronym}[Visual Studio] %Längster Begriff
\setlength{\itemsep}{-\parsep}
	\acro{.apk}{Android Package}
	\acro{ACL}{Access Control Lists}
	\acro{AES}{Advanced Encryption Standard}
	\acro{AJAX}{Asynchronous JavaScript and XML}	
	\acro{ANR}{Application Not Responding}
	\acro{API}{Application Programming Interface}
	\acro{CORS}{Cross-origin resource sharing}
	\acro{CRUD}{Create Read Update Delete}	
	\acro{CSS}{Cascading Style Sheet}
	\acro{DLL}{Dynamic Link Library}
	\acro{DRAM}{Dynamischer \ac{RAM}}
	\acro{HTML}{Hypertext Markup Language}	
	\acro{HTTP}{Hypertext Transfer Protocol}	
	\acro{HTTPS}{Hyper Text Transfer Protocol Secure}	
	\acro{IDE}{Integrated Development Environment}
	\acro{JSON}{JavaScript Object Notation}	
	\acro{JWT}{\ac{JSON} Web Token}	
	\acro{MSSQL}{Microsoft SQL Server}
	\acro{MVC}{Model-View-Controller}	
	\acro{MVVM}{Model-View-ViewModel}
	\acro{OR-Mapper}{objekt-relationaler Mapper}
	\acro{PCL}{Portable Class Library}
	\acro{RAM}{Random Access Memory}
	\acro{RBAC}{Role Based Access Control}	
	\acro{REST}{Representational State Transfer}	
	\acro{SPA}{Single Page Application}
	\acro{SRAM}{Statischer \ac{RAM}}
	\acro{TCP}{Transmission Control Protocol}
	\acro{URI}{Uniform Resource Identifier}		
	\acro{URL}{Uniform Resource Locator}		
	\acro{Visual Studio}{Microsoft Visual Studio 2015 Community Edition}		
	\acro{VM}{Virtuelle Maschine}
	\acro{Web-App}{mobil-optimierte Webseite}
	\acro{XML}{Extensible Markup Language}
\end{acronym}



%% Kopf und Fusszeilen====================================================
%%% Doc: ftp://tug.ctan.org/pub/tex-archive/macros/latex/contrib/koma-script/scrguide.pdf
\usepackage[%
   automark,         % automatische Aktualisierung der Kolumnentitel
   %nouppercase,      % Grossbuchstaben verhindern
   %markuppercase    % Grossbuchstaben erzwingen
   %markusedcase     % vordefinierten Stil beibehalten
   %komastyle,       % Stil von Koma Script
   %standardstyle,   % Stil der Standardklassen
]{scrpage2}



%% UeberSchriften (Chapter und Sections) =================================
% -- Ueberschriften komlett Umdefinieren --
%%% Doc: ftp://tug.ctan.org/pub/tex-archive/macros/latex/contrib/titlesec/titlesec.pdf
\usepackage{titlesec}

% -- Section Aussehen veraendern --
% --------------------------------
%% -> Section mit Unterstrich
% \titleformat{\section}
%   [hang]%[frame]display
%   {\usekomafont{sectioning}\Large}
%  {\thesection}
%   {6pt}
%   {}
%   [\titlerule \vspace{0.5\baselineskip}]
% --------------------------------

% -- Chapter Aussehen veraendern --
% --------------------------------
\titleformat{\chapter}[block]	% {command}[shape]
  {\usekomafont{chapter}\huge\sffamily\bfseries}	% format
  {   										% label
  {\thechapter.} \filright%
  }%}
  {1pt}										% sep (from chapternumber)
  {\vspace{0.5pc} \filright}   % {before}[after] (before chaptertitle and after)
  [\vspace{0.5pc} \filright {}]

% \titleformat{\chapter}[]%
%    {\usekomafont{chapter}\huge\sffamily\bfseries}%
%    {\thechapter}%
%    {1em}%
%    {}%


\usepackage{rotating}

%Literatur-Einstellungen

%vielleicht Layout anpassen...
\usepackage{jurabib}

\jurabibsetup{
	commabeforerest,
	ibidem=strict,
	citefull=first,
	see,
	titleformat={colonsep,all},
}

%Kurzzitat-Schreibweise
%\usepackage[%
%authorformat={smallcaps,year},%
%round,%
%commabeforerest%
%]{jurabib}
%\renewcommand{\jbcitationyearformat}[1]{\unskip, #1}

\renewcommand*{\jbauthorfont}{\textsc}
\renewcommand*{\biblnfont}{\scshape\textbf}
\renewcommand*{\bibfnfont}{\normalfont\textbf}


% Quotes =================================================================
%% Doc: ftp://tug.ctan.org/pub/tex-archive/macros/latex/contrib/csquotes/csquotes.pdf
% Advanced features for clever quotations
\usepackage[%
   babel,            % the style of all quotation marks will be adapted
                     % to the document language as chosen by 'babel'
   german=quotes,		% Styles of quotes in each language
   %german=guillemets,
   english=british,
   french=guillemets
]{csquotes}
\usepackage{floatflt}

\usepackage{wrapfig}

%\usepackage{subfigure}

\usepackage{blindtext}

\usepackage{listings}
\definecolor{lila}{RGB}{112, 6, 147}
\definecolor{kommentgreen}{RGB}{5,132,71}
\definecolor{grey}{RGB}{242,242,242}  
\definecolor{darkgreen}{named}{green}
\definecolor{darkblue}{named}{blue}
\definecolor{lightblue}{RGB} {63,95,191}
\definecolor{darkred}{named}{red}
\definecolor{grau}{named}{gray}
\definecolor{fh_orange}{rgb}{0.953,0.201,0}
\definecolor{fh_grau}{rgb}{0.76,0.75,0.76}

\definecolor{listinggray}{gray}{0.9}
\definecolor{lbcolor}{rgb}{0.9,0.9,0.9}

\lstset{
	tabsize=3,
	float=tbph,
	frame=single,
	extendedchars,
	breaklines=true,
	basicstyle=\fontsize{9pt}{9pt}\selectfont,
	columns=flexible, %ist notwendig, damit man Quelltext aus den Listings kopieren kann
	numbers=left, 
	numberstyle=\color{black},
	captionpos=b,
	aboveskip=7mm,
	backgroundcolor=\color{grey}
}

\lstdefinestyle{java}
{
	language=Java,
	keywordstyle=\color{lila},  	% underlined bold black keywords 
	identifierstyle=\color{blue}, 
	commentstyle=\color{kommentgreen}, % white comments 
	stringstyle=\color{black},
}

\lstdefinestyle{xml}
{
	language=xml,
	basicstyle=\fontsize{9pt}{9pt}\selectfont\color{kommentgreen},
	keywordstyle=\color{lila},  	% underlined bold black keywords 
	%Hier können bei Bedarf noch weitere Keywords eingetragen werden
	keywords={name, value, version, encoding, id, type, xmlns:xsi, ref, namespace},
	identifierstyle=\color{black},  
	stringstyle=\color{blue},  
	commentstyle=\color{lightblue},
	morecomment=[s]{<!--}{-->},
	rulecolor=\color{black}
}


\definecolor{bluekeywords}{rgb}{0,0,1}
\definecolor{greencomments}{rgb}{0,0.5,0}
\definecolor{redstrings}{rgb}{0.64,0.08,0.08}
\definecolor{xmlcomments}{rgb}{0.5,0.5,0.5}
\definecolor{types}{rgb}{0.17,0.57,0.68}

\lstdefinestyle{sharpc}{
language=[Sharp]C, 
frame=lines, % Oberhalb und unterhalb des Listings ist eine Linie
showspaces=false,
showtabs=false,
breaklines=true,
showstringspaces=false,
breakatwhitespace=true,
escapeinside={(*@}{@*)},
commentstyle=\color{greencomments},
morekeywords={partial, var, value, get, set},
keywordstyle=\color{bluekeywords},
stringstyle=\color{redstrings},
basicstyle=\ttfamily\small,
escapechar=|
}

\usepackage{multicol}

\usepackage{nameref}

\usepackage{hyperref}
\hypersetup{breaklinks=true}
\hypersetup{colorlinks=true,linkcolor=black,urlcolor=black,citecolor=black}
%\hypersetup{frenchlinks}	% Use small caps instead of color for links
%\hypersetup{pdfpagemode=FullScreen}
%\hypersetup{pdfstartpage=3}
%\hypersetup{pdfstartview=Fit}


% Tabellen ueber mehere Seiten
% ----------------------------
%%% Doc: ftp://tug.ctan.org/pub/tex-archive/macros/latex/contrib/carlisle/ltxtable.pdf
% \usepackage{ltxtable} % Longtable + tabularx
                        % (multi-page tables) + (auto-sized columns in a fixed width table)
% -> nach hyperref laden
%\usepackage{ltxtable}
%\usepackage{longtable}
\usepackage{tabulary}


% Schusterjunge und Hurenkinder verhindern
\clubpenalty=1000
\widowpenalty=1000
\displaywidowpenalty=1000

% Trennen von Bindestrich oder so ...
%\defaulthyphenchar=127

\usepackage{ucs}               % Extended UTF-8 input encoding
\usepackage[utf8x]{inputenc}   % trans­lates var­i­ous stan­dard and other in­put en­cod­ings into a ‘LATEX in­ter­nal lan­guage’
\usepackage[T1]{fontenc}
\usepackage{alltt}
\usepackage{marvosym}
\usepackage{fancybox}
\usepackage[hang,small,bf]{caption}
\usepackage{float} 
\usepackage{multirow}
\usepackage[nomain,acronym,xindy,toc]{glossaries} 
\makeglossaries
\usepackage[xindy]{imakeidx}
\makeindex