\makenoidxglossaries

\newglossaryentry{OR-Mapper}{
name={OR-Mapper},
description={Ein Framework zur Überführung von relationalen Tupeln in objekt-orientierte Objekte}
}
\newglossaryentry{App}{
name={App},
description={Kleine Programme/Applikationen für mobile Endgeräte}
}
\newglossaryentry{UserStory}{
name={User Story},
description={Ein Anwendungsfall einer Software. Hierbei ist es möglich, dass die Abarbeitung über mehrere Oberflächen und Aktionen umgesetzt werden kann}
}
\newglossaryentry{responsiv}{
name={responsiv},
description={Eine Webseite, welche responsiv gestaltet wird, passt ihr aussehen abhängig vom aufrufenden Endgerät an. Diese Technik kommt besonders Geräten mit kleineren Displays wie Smartphones oder Tablets zu Gute}
}
\newglossaryentry{SRAM}{
name={SRAM}, 
description={Ein SRAM ist ein statischer RAM, der eine sehr schnelle Zugriffszeit garantiert und Daten mit einem sehr geringen Stromaufwand speichern kann}
}
\newglossaryentry{DRAM}{
name={DRAM}, 
description={Ein DRAM ist ein dynamischer RAM, der sehr langsam und kostengünstig ist}
}
\newglossaryentry{Request}{
name={Request}, 
description={Ein Request ist ein Aufruf an einen Knoten. Meist ein Aufruf an einen Server im Internet}
}
\newglossaryentry{Browser}{
name={Browser}, 
description={Ein Browser ist ein Programm zum Anzeigen von Internetseiten aus dem World Wide Web}
}
\newglossaryentry{HTML5}{
name={HTML5}, 
description={HTML5 meint die fünfte Version von \ac{HTML}. Es bietet beispielsweise besondere Auszeichnungen für Kopf-, Fuß- und Navigationsbereiche}
}
\newglossaryentry{CSS3}{
name={CSS3}, 
description={Die aktuelle Version der Stylesheet-Sprache \ac{CSS}}
}
\newglossaryentry{Javascript}{
name={Javascript}, 
description={Eine Programmierbrache, welche von Browsern für die clientseitige Ausführung von Code benutzt wird. Dadurch kein einer Webseite dynamisches Verhalten hinzugefügt werden.}
}
\newglossaryentry{Android}{
name={Android}, 
description={Mobiles Betriebssystem der Firma Google}
}
\newglossaryentry{iOS}{
name={iOS}, 
description={Mobiles Betriebssystem der Firma Apple}
}
\newglossaryentry{XCode}{
name={XCode}, 
description={Eine von Apple entwickelte Entwicklungsumgebung zu Erstellung und Wartung von Anwendungen für die Betriebssysteme von Mac OS X, iOS und watchOS }
}
\newglossaryentry{C-sharp}{
name={C\#}, 
description={Eine von Microsoft entwickelte hohe Programmiersprache zur Umsetzung von Applikationen auf Grundlage des .Net-Frameworks}
}
\newglossaryentry{Seperation-of-Concerns}{
name={Separation of Concerns}, 
description={Bei diesem Design-Pattern wird darauf geachtet, dass sich Aufgabenbereiche nicht überschneiden. Jeder Teil eines Problems soll durch einen eigenen Teil gelöst werden}
}
\newglossaryentry{MSEF}{
name={Microsoft Entity Framework}, 
description={Ein OR-Mapper von \textit{Microsoft}. Er ist besonders gut zur Kommunikation mit \textit{\ac{MSSQL}} geeignet}
}

% Nutze \gls als Referenz fuer ein Glossar-Wort oder \glspl um die Plural-Form auszugeben, diese muss mit angegeben werden: 

% die gesamte Beschriebung findet man unter http://ctan.space-pro.be/tex-archive/macros/latex/contrib/glossaries/glossariesbegin.pdf