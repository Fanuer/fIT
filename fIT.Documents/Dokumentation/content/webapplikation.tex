\chapter{Realisierung der clientseitigen Implementierung als Webapplikation}
\label{cha:web-app}

\section{Definition einer Single Page Application}
\label{sec:Definition-SPA}

\section{AngularJs}
\label{sec:AngularJs}

\subsection{MVC}
\label{ssec:SPA-MVC}

\subsection{Services}
\label{ssec:SPA-Services}

\subsection{Promises}
\label{ssec:SPA-Promises}

\subsection{Routing}
\label{ssec:SPA-Routing}

\section{Umsetzung}
\label{sec:SPA-Umsetzung}

\subsection{Layout mit Twitter Bootstrap}
\label{ssec:SPA-twitter-bootstrap}

\begin{itemize}
\item Was ist das 
\item Vorteile: responsives Verhalten 
\end{itemize}

\subsection{Online-Check}
\label{ssec:Online-Check}

\subsection{Herausforderung statusloses Protokol Http}
\label{ssec:statusloses-http}

\begin{itemize}
\item Login
\end{itemize}

\section{CachedHttpService mit IndexedDB}
\label{sec:CachedHttpService}

\subsection{Exkurs IndexedDB}
\label{ssec:Exkurs-IndexedDB}

\subsection{Http-Verbs}
\label{ssec:Http-Verbs}

Umsetzung von Caching auf basis der Http-Verben statt einer konkreten implementierung für jede entity

\subsection{Synchronisation zwischen Server und SPA}
\label{ssec:Sync-SPA}

\section{Herausforderungen}
\label{sec:Herausforderungen-SPA}

\begin{itemize}
\item IndexedDB nicht voll implementiert
\item Code vollständig einsehbar: Probleme mit sensiblen Daten
\end{itemize}
