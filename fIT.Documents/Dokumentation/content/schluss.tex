\chapter{Fazit}
\label{fazit}
Anfangs wurde ein Überblick über die Aufgabe gegeben und die Systemarchitektur vorgestellt. Darauf aufbauend wurden die verwendeten Technologien erörtert und die Umsetzungen der beiden Applikationen bis zu einem festgelegten Punkt dokumentiert. Die Implementierung des Servers wurde zusätzlich betrachtet. Die anschließende Gegenüberstellung der Apps hat ergeben, dass die Weiterentwicklung der Android-App in Hinsicht auf Sicherheit als einzige Lösung gesehen werden kann.\\
Deshalb wurde für diese Applikation ein zweiter Meilenstein begonnen, um die Anforderungen an den umzusetzenden Prototypen zu implementieren. 

\section{Ziele / Ergebnisse}
\label{sec:ziele-ergebnisse}
Rückblickend kann das Projekt als ein großer Erfolg gesehen werden. Die eingangs formulierten Ziele konnten vollständig umgesetzt werden. Zudem wurden teilweise noch Funktionen umgesetzt, die über das formulierte Ziel hinaus gehen.\\
Zusammenfassend ist es nun mit dem Prototypen möglich den kompletten Funktionsumfang der nativen Android-App auch im \textit{Oflline}-Modus nutzen zu können. 

\section{Erkenntnisse}
\label{sec:erkenntnisse}
Der Erkenntnisgewinn dieser Arbeit ist beträchtlich. Es wurden ausschließlich unbekannte und für die Autoren neue Technologien verwendet. Die Einarbeitung geschah in den meisten Fällen reibungslos, war jedoch auch ein vorher unbekanntes Risiko, welches ein Problem in der Umsetzung hätte verursachen können.\\
Des Weiteren ist augenscheinlich, dass der Umgang mit den mobilen System, sei es Android über Xamarin oder Webapplikation über AngularJS, vertieft wurde. Darüber hinaus wurde mit dem dahinter fungierenden Server eine Einheit geschaffen, die die Erweiterung des Prototypen auch für eine größere Menge Benutzer ermöglicht.
\section{Ausblick}
\label{sec:ausblick}
Die Grundlage für den Ausbau dieses Projektes zu einer marktreifen App ist allemal gegeben. Der dahinter liegende Server ist stark genug, um eine größere Last an Anfragen zu bewältigen. Zum Ausbau dieses Projekts muss die Android-App weiterentwickelt werden. Die Umsetzung aller wichtiger Funktionen wurde jeweils an mindestens einem Beispiel im Messeprototypen dargestellt und muss demnach nur noch auf die fehlenden Funktionalitäten übertragen werden.\\
Durch den nun tieferen Einblick in die Technologien sollte sich dieser Aufwand in Grenzen halten.