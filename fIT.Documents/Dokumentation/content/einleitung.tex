\chapter{Einleitung}
\label{cha:einleitung}
In diesem Kapitel wird das grundlegende Problem und die daraus resultierende Aufgabenstellung erläutert.

\section{Problemstellung}
\label{sec:problemstellung}

\textbf{TODO: Rausnehmen}
Beschreibung: Wer was gemacht hat -> Übersicht je Kapitel

Mobile Applikationen sind im täglichen Leben allgegenwärtig.
 
Eine Herausforderung bei diesen Anwendungen ist es, dass sie verlässlich funktionieren müssen, da ansonsten ein Schaden auftritt, welcher sogar lebensbedrohlich- oder zumindest finanziell sein kann.
Da dieses Problem in unterschiedlichen Anwendungen  immer wieder auftaucht, ist es sinnvoll, hierfür einen generischen Ansatz anzubieten. 

Für mobile Endgeräte können zwei unterschiedliche Lösungsansätze verfolgt werden: 
\begin{itemize}
\item die Entwicklung nativer Apps und
\item die Entwicklung mobiler Webseiten.
\end{itemize}
Diese beiden Lösungsansätze sollen unter dem Aspekt der Verlässlichkeit gegenübergestellt werden.

Im Zuge der Arbeit soll geprüft werden, ob Caching besser über eine native App oder über eine responsive Web-Applikation umgesetzt werden können. 
Hierbei sollen die Vor- und Nachteile von mobilen Webapplikationen gegenüber nativen Apps beleuchtet werden.


In diesem Unterkapitel sollten folgende Punkte behandelt werden:
\begin{itemize}
	\item Was ist das Problem?
	\item Problemgeschichte?
\end{itemize}

\section{Zielsetzung}
\label{sec:zielsetzung}
\begin{itemize}
	\item Was soll mit der Arbeit erreicht werden? Welche Ziele werden angestrebt? Möglichst kurz und präzise geplante Ergebnisse umreißen. $/rightarrow$ Daran werden Ihre Resultate am Ende gemessen!

\end{itemize}


\section{Vorgehensweise}
\label{sec:vorgehensweise}
\begin{itemize}
	\item Wie wird vorgegangen, um das Ziel zu erreichen?
	\item Warum ist die Arbeit so gegliedert, wie sie gegliedert ist?
	\item Welche Aspekte werden nicht behandelt \textbf{und} warum?
\end{itemize}