\makenoidxglossaries

\newglossaryentry{OR-Mapper}{
name={OR-Mapper},
description={Ein Framework zur Überführung von relationalen Tupeln in objekt-orientierte Objekte}
}
\newglossaryentry{App}{
name={App},
description={Kleine Programme/Applikationen für mobile Endgeräte}
}
\newglossaryentry{UserStory}{
name={User Story},
description={Ein Anwendungsfall einer Software. Hierbei ist es möglich, dass die Abarbeitung über mehrere Oberflächen und Aktionen umgesetzt werden kann}
}
\newglossaryentry{responsiv}{
name={responsiv},
description={Eine Webseite, welche responsiv gestaltet wird, passt ihr aussehen abhängig vom aufrufenden Endgerät an. Diese Technik kommt besonders Geräten mit kleineren Displays wie Smartphones oder Tablets zu Gute}
}
\newglossaryentry{SRAM}{
name={SRAM}, 
description={Ein SRAM ist ein statischer RAM, der eine sehr schnelle Zugriffszeit garantiert und Daten mit einem sehr geringen Stromaufwand speichern kann}
}
\newglossaryentry{DRAM}{
name={DRAM}, 
description={Ein DRAM ist ein dynamischer RAM, der sehr langsam und kostengünstig ist}
}
\newglossaryentry{Request}{
name={Request}, 
description={Ein Request ist ein Aufruf an einen Knoten. Meist ein Aufruf an einen Server im Internet}
}
\newglossaryentry{Browser}{
name={Browser}, 
description={Ein Browser ist ein Programm zum Anzeigen von Internetseiten aus dem World Wide Web}
}
\newglossaryentry{HTML5}{
name={HTML5}, 
description={HTML5 meint die fünfte Version von \ac{HTML}. Es bietet beispielsweise besondere Auszeichnungen für Kopf-, Fuß- und Navigationsbereiche}
}
\newglossaryentry{CSS3}{
name={CSS3}, 
description={Die aktuelle Version der Stylesheet-Sprache \ac{CSS}}
}
\newglossaryentry{Javascript}{
name={Javascript}, 
description={Eine Programmierbrache, welche von Browsern für die clientseitige Ausführung von Code benutzt wird. Dadurch kein einer Webseite dynamisches Verhalten hinzugefügt werden.}
}
\newglossaryentry{Android}{
name={Android}, 
description={Mobiles Betriebssystem der Firma Google}
}
\newglossaryentry{iOS}{
name={iOS}, 
description={Mobiles Betriebssystem der Firma Apple}
}
\newglossaryentry{XCode}{
name={XCode}, 
description={Eine von Apple entwickelte Entwicklungsumgebung zu Erstellung und Wartung von Anwendungen für die Betriebssysteme von Mac OS X, iOS und watchOS }
}
\newglossaryentry{C-sharp}{
name={C\#}, 
description={Eine von Microsoft entwickelte hohe Programmiersprache zur Umsetzung von Applikationen auf Grundlage des .Net-Frameworks}
}
\newglossaryentry{Seperation-of-Concerns}{
name={Separation of Concerns}, 
description={Bei diesem Design-Pattern wird darauf geachtet, dass sich Aufgabenbereiche nicht überschneiden. Jeder Teil eines Problems soll durch einen eigenen Teil gelöst werden}
}
\newglossaryentry{MSEF}{
name={Microsoft Entity Framework}, 
description={Ein OR-Mapper von \textit{Microsoft}. Er ist besonders gut zur Kommunikation mit \textit{\ac{MSSQL}} geeignet}
}
\newglossaryentry{Repository}{
name={Repository}, 
description={Das Repository-Entwurfsmuster kapselt die Datenschicht von der Applikationsschicht. Es bietet Schnittstellen an, mit dem die Applikationsschicht auf die Datenschicht zugreifen kann. Dadurch ist es später leichter, eine Datenschicht auszutauschen oder mehrere Datenquellen anzubinden}
}
\newglossaryentry{Factory}{
name={Factory}, 
description={Das Factory-Entwurfsmuster ist ein Erzeugungsmuster. Eine Klasse, welche das Factory-Pattern benutzt, stellt Methoden zur Erzeugung von Objekten bereit. Diese Klasse ist daraufhin alleinig für die Erstellung der unterstützten Objekte verantwortlich. Der Vorteil liegt in der zentralen Anlaufstelle. Dadurch kann bestimmt werden, welche Programmstelle neue Objekte erzeugen soll und in welcher Art diese erzeugt werden}
}
\newglossaryentry{NuGet}{
name={NuGet}, 
description={Eine Paketverwaltung für das .Net-Framework. Es erlaubt das Hinzufügen, Aktualisieren und Entfernen von Komponenten und deren Abhängigkeiten}
}
\newglossaryentry{TCP}{
name={TCP}, 
description={TCP steht für Transmission Control Protocol. Dies ist ein verbindungsorientiertes Übertragungsprotokoll zum bidirektionalen Datentransport zwischen Computern}
}
\newglossaryentry{JSONP}{
name={JSONP}, 
description={JSONP steht für \ac{JSON} mit Padding. Es ist eine Möglichkeit zur übertragung von JSON-Daten über Domaingrenzen hinweg}
}
\newglossaryentry{Polyfills}{
name={Polyfills}, 
description={Ein Codebaustein, welcher aktuelle Web Techniken in älteren Browsern verfügbar macht}
}
\newglossaryentry{Fat-Client}{
name={Fat-Client}, 
description={Art eines Clients in einer Server-Client-Architektur. Dieser zeichnet sich dadurch aus, dass viele genutzte Funktionalitäten dezentral auf den Clients vorhanden sind. Dadurch sind weniger Anfragen an den Server nötig, was die Unabhängigkeit des Clients erhöht}
}
\newglossaryentry{Makro}{
name={Makro},
description={Kapselung von Code-Anweisungen zur Umsetzung einer Aufgabe}
}

\newglossaryentry{Markup}{
name={Markup},
description={Auszeichnung von Text, sodass zusätzliche Semantik geschaffen wird. Beispiele hierfür sind \ac{HTML} und \ac{XML}}
}

\newglossaryentry{VMMV}{
name={VMMV},
description={Model-View-ViewModel-Muster. Eine Abwandlung des \ac{MVC}-Musters. Hierbei werden die Daten aus dem Model in ein ViewModel überführt, bevor dieses an die View übergeben wird. Dadruch erhält man eine höhere Kontrolle, welche und wie Daten vom Model an die View weitergegeben werden}
}

\newglossaryentry{RBAC}{
name={RBAC},
description={Steht für Rolebased Access Control. Ein Konzept, bei der dem Nutzer Rollen zugewiesen werden. Die Rollen sind für bestimmte Funktionalitäten autorisiert. Somit kann leichter eine Zuordnung von Funktionalitäten zu einer Nutzgruppe administriert werden}
}

\newglossaryentry{OPTIONS}{
name={OPTIONS},
description={Ein \ac{HTTP}-Verb, welches zum Abruf von Meta-Daten zur Kommunikation mit dem Server genutzt wird}
}

\newglossaryentry{monolithisch}{
name={monolithisch},
description={Ein Betriebssystem wird als monolithisches bezeichnet, wenn es sich nicht nur um die Speicher- und Prozessverwaltung kümmert, sondern auch die benötigten Treiber zur Verfügung stellt.}
}

\newglossaryentry{Linux}{
name={Linux},
description={Linux ist ein quelloffenes Betriebssystem, welches durch eine Community weiterentwickelt wird.}
}

\newglossaryentry{Dirty Read}{
name={Dirty Read},
description={Bezeichnet einen Zugriff auf eine Ressource, welche sich noch in Bearbeitung durch eine andere Komponente befindet. Somit wird ein altes Datum abgerufen.}
}
\newglossaryentry{NoSQL}{
name={NoSQL},
description={Steht für \textit{Not only SQL}. NoSQL-Datenbanken besitzen keine klassischen relationalen Tabelleschemata. Dies erlaubt eine leichtere Speicherung von dynamisch strukturierten Daten. Es ist als ein strukturierter Datenspeicher zu verstehen, bei denen auf die Einträge per Index zugegriffen werden kann}
}

\newglossaryentry{Singleton}{
name={Singleton},
description={Ein Entwurfsmuster, dessen Implementierung gewährleistet, dass ein Objekt einer Klasse nur genau ein Mal erzeugt wird. Wird dieses Objekt erneut benötigt, wird das bereits erzeugte Objekt zurückgegeben}
}

% Nutze \gls als Referenz fuer ein Glossar-Wort oder \glspl um die Plural-Form auszugeben, diese muss mit angegeben werden: 

% die gesamte Beschriebung findet man unter http://ctan.space-pro.be/tex-archive/macros/latex/contrib/glossaries/glossariesbegin.pdf