\makenoidxglossaries

\newglossaryentry{OR-Mapper}{name={OR-Mapper},description={Ein Framework zur Überführung von relationalen Tupeln in objekt-orientierte Objekte}}
\newglossaryentry{App}{name={App},description={Kleine Programme/Applikationen für mobile Endgeräte}}
\newglossaryentry{UserStory}{name={User Story},description={Ein Anwendungsfall einer Software. Hierbei ist es möglich, dass die Abarbeitung über mehrere Oberflächen und Aktionen umgesetzt werden kann}}
\newglossaryentry{responsiv}{name={responsiv},description={Eine Webseite, welche responsiv gestaltet wird, passt ihr aussehen abhängig vom aufrufenden Endgerät an. Diese Technik kommt besonders Geräten mit kleineren Displays wie Smartphones oder Tablets zu Gute.}}


% Nutze \gls als Referenz fuer ein Glossar-Wort oder \glspl um die Plural-Form auszugeben, diese muss mit angegeben werden: 

% die gesamte Beschriebung findet man unter http://ctan.space-pro.be/tex-archive/macros/latex/contrib/glossaries/glossariesbegin.pdf