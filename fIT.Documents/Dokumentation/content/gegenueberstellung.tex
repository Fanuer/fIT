\chapter{Gegenüberstellung der clientseitigen Implementierungen}
\label{cha:gegenueberstellung}
Ziel dieses Kapitels ist die Gegenüberstellung der Erkenntnisse zur Entwicklung einer verlässlichen mobilen Applikation. Hierbei wird das neu erlangte Wissen zur Umsetzung einer Applikation als \ac{SPA} und als native \gls{App} bewertet, sodass mit der vorteilhafteren der beiden Optionen der Messeprototyp umgesetzt werden kann. \\
Das Kapitel schließt auch gleichzeitig die Entwicklung des Meilensteins 1 ab.

\section{Umsetzung als Web App}
\label{sec:gegenueberstellung-SPA}
Als Erstes sollen die Vor- und Nachteile der Umsetzung des Clients als Single Page Application aufgezeigt werden.

\subsection{Vorteile}
\label{sec:vorteile-SPA}
Bei der Umsetzung des Clients als Web Applikation zeigen sich die Vorteile besonders in der Umsetzung der Oberfläche. \\
Durch die Nutzung aktueller Webtechniken und geeigneter Frameworks lässt sich sehr leicht ein einheitliches Aussehen schaffen, welches für verschiedene Anzeigegrößen optimiert ist. Hierbei ist man nicht nur auf mobile Endgeräte beschränkt, sondern erhält zusätzlich eine Webseite, die eine Desktop-Anwendung ersetzen kann. Auch die Umsetzung der Business-Logik konnte ohne großen Einarbeitungsaufwand bewerkstelligt werden. Dabei fällt auf, dass durch das Voranschreiten von \gls{HTML5} viele Funktionen, welche vor einigen Jahren nur durch Desktop Applikationen umgesetzt werden konnten, heute schon problemlos im Browser abbildbar sind. Hierbei zeigten sich aber auch die Schwächen einer Umsetzung als Web Applikation. 

\subsection{Nachteile}
\label{sec:nachteile-SPA}
Wie bereits erwähnt sind viele, jedoch noch nicht alle, Techniken für den Browser umgesetzt. So ist die Umsetzung der \textit{IndexedDB} für \textit{iOS} und \textit{Microsoft}-Geräte noch sehr fehleranfällig\footcite{online:caniuse:indexedDB}. \\
In diesem Punkt spiegelt sich auch das größte Problem bei der Umsetzung von Web Anwendungen wieder: Unterschiedliche Browser implementieren einige \ac{API}s anders oder teilweise auch gar nicht, sodass viel Entwicklungszeit für das Anpassen der Funktionen und Oberflächen für die verschiedenen Browser genutzt werden muss. Wenn es nun so ist, dass Kernkomponenten, wie in unserem Fall die \textit{IndexedDB}, in einigen wichtigen Browsern, wie beispielsweise dem Safari-Browser unter \gls{iOS} (Nutzung Version 8 bei 7.33\% (Stand 31.08.2015)\footcite{online:caniuse:indexedDB}), nicht ausreichend unterstützt werden, ist die Umsetzung dieses Teilaspektes für den produktiven Einsatz fast unmöglich. \\
Ein weiterer Nachteil ergibt sich aus der Nutzung von AngularJS. Da die gesamte Datenaufbereitung mit der Authentifizierung und dem Routing auf Seiten des Clients passiert, können die lokal gespeicherten Daten, mit Hilfe der Entwicklungswerkzeuge des Browsers, einfach ausgelesen werden. Deswegen wäre es unter Sicherheitsaspekten fahrlässig, die hier vorgestellte Implementierung der Authentifizierung (siehe Kapitel \ref{ssec:statusloses-http}) ohne weitere Sicherheitsmaßnahmen produktiv einzusetzen.

\section{Umsetzung als native App}
\label{sec:gegenueberstellung-native-app}
Da nun die \ac{SPA} mit den Vor- und Nachteilen vorgestellt wurde, sollen auch die verschiedenen Aspekte bei der Umsetzung als native Applikation vorgestellt werden. 

\subsection{Vorteile}
\label{sec:vorteile-native-app}
Die Vorteile einer nativen Applikation liegen besonders in dem großen Funktionsumfang. Dieser kann alle bereitgestellten Funktionen des Betriebssystems ausnutzen. Dazu zählen das \textit{Threading} (siehe Kapitel \ref{ssec:android-prozesse-threads}) und interne Aufrufe über \textit{Services}, welche zum Beispiel zum Versenden von E-Mails verwendet werden können. Zudem können Daten persistent, auch über die Dauer einer \textit{Session} hinaus, auf dem Gerät gespeichert werden (siehe Kapitel \ref{ssec:nat-db}).\\
Unter Sicherheitsaspekten ist die native \gls{App} einer \ac{Web-App} vorzuziehen, da Daten nur mit guten technischen Kenntnissen ausgelesen werden können. Sind die Daten darüber hinaus lokal verschlüsselt, entsteht ein noch höherer Grad an Sicherheit.\\
Weiterhin ist die gesamte Logik der Applikation für den Endanwender nicht sichtbar. Dadurch ist eine Manipulation der Daten wesentlich schwieriger, als bei einer \ac{SPA}. Diese sind nur über aufwendige programmatische Eingriffe möglich.

\subsection{Nachteile}
\label{sec:nachteile-native-app}
Die Umsetzung der nativen \gls{App} beansprucht viel Zeit und ein grundlegendes Knowhow über die Funktionsweise des zu unterstützenden Betriebssystems.\\
Darin liegt auch noch ein weiteres Problem: Um eine große Marktabdeckung mit einer nativen \gls{App} zu erreichen, benötigt man mindestens eine \gls{iOS}- und eine \gls{Android}-Applikation. Erst dann hat man Zugang zu über 95\% der Smartphone-Nutzer in Deutschland.\footcite{Statista-SmartphoneVerteilung}\\
Die Entwicklung für zwei Systeme kann daraufhin in zwei Möglichkeiten umgesetzt werden: Zum einen könnten native Apps in den jeweiligen Programmiersprachen entwickelt werden. Zum anderen kann eine Multiplattform-Lösung, wie Xamarin-Platform es ist, eingesetzt werden. Dabei beschränkt sich der Funktionsumfang jedoch hauptsächlich auf die grundlegenden Funktionen. Auf Spezialfunktionen eines bestimmten Systems kann nur  zugegriffen werden, wenn man eine Applikation exklusiv für eine spätere Endplattform erstellt.Weiterhin ist das Erstellen von Oberflächen aufwendiger als bei einer Single Page Application.\\
Zudem müssen native \glspl{App} direkt auf das Endgerät geladen und installiert werden und können nicht einfach über das Internet aufgerufen werden. 

\section{Fazit aus Meilenstein 1}
\label{sec:gegenueberstellung-fazit}
Zusammenfassend kann festgehalten werden, dass die Nachteile der Single Page Application dahingehend überwiegen. Besonders die aktuell unzureichend implementierte Datenbank in einigen mobilen Browsern lässt es aktuell nicht zu, dass eine mobile Applikation für mehrere mobile Endgeräte entwickelt werden kann. Darüber hinaus kann die Sicherheit der Daten - besonders in Verbindung mit Vitaldaten - aktuell nicht vollständig gewährleistet werden kann. Diese Anforderung kann aktuell nur von einer nativen \gls{App} zufriedenstellend geleistet werden. 
Ändert sich der Zustand der Implementierung, müsste diese Evaluation neu durchgeführt werden. \\
Somit wird im weiteren Vorgehen der Prototyp der nativen \gls{App} zu einem Messeprototyp weiterentwickelt, da die Argumente gegen die Weiterentwicklung der \ac{Web-App} bei einer native \gls{App} nicht gegeben sind. Die Nachteile einer nativen Applikation sind darüber hinaus für die Weiterentwicklung zum Messeprototyp vernachlässigbar.
