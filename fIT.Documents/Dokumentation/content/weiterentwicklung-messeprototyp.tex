\chapter{Weiterentwicklung eines Clients zu einem Messeprototyp}
\label{cha:weiterentwicklung-messeprototyp}
Nach der Entscheidung für die Weiterentwicklung der nativen Android-App, begann die Planung der möglichen Erweiterungen. Dabei wurden besonders Performance- und Stabilitätsaspekte in den Vordergrund gestellt. Darüber hinaus sollte aber auch die Oberfläche einem Messeprototypen entsprechend verbessert werden und Funktionen, die aus den Kann-Kriterien des Pflichtenheftes entspringen, umgesetzt werden, um einen größeren Funktionsumfang präsentieren zu können.

\section{Anpassungen der Ablauflogik}
\label{sec:anpassungen-ablauflogik}
Die Ablauflogik der nativen App wurde weitestgehend beibehalten, da die Planung im Vorfeld schon eine komplette \textit{User-Story} vorgesehen hat. So muss man sich zuerst anmelden, um dann durch die Trainingspläne und Übungen navigieren zu können und abschließend die Möglichkeit hat ein Training einzutragen. Demnach sind in diesem Sinne keine Verbesserungen oder Änderungen sinnvoll.\\
Aufgrund dessen wurden Anpassungen vorgenommen, die nach Außen nicht sichtbar sind, die Stabilität und die Leistungsfähigkeit der App aber zu einem sehr großen Teil verbessert haben. So wurde die Synchronisation aus den Methoden der \textit{Get}- und \textit{Post}-Abfragen extrahiert und zentralisiert (Vergleich dazu in Kapitel \ref{ssec:cache-unsere-funktionsweise}).\\
Die Funktionalität des \textit{Caches} wurde nur hinsichtlich der Synchronisation mit dem Server angepasst.  Darüber hinaus wurden keine Änderungen vorgenommen und das Eintragen von Daten erfolgt jeweils beim Abrufen von Server-Daten, sowie beim Übertragen von Daten zum Server.
%Sequenzdiagramm-RecentOnline
\begin{figure}[h]
\centering
\includegraphics[width=\linewidth]{content/images/fITNat-RecentOnline}
\caption{Squenzdiagramm \textit{RecentOnline}}
\label{pic:nat-RecentOnline}
\end{figure}
Die Abbildung \ref{pic:nat-RecentOnline} verdeutlicht den neuen Ablauf des Synchronisierens. Bedingung zum Start des Synchronisierens ist die Tatsache, dass die Applikation aktuell eine Verbindung zum Server besitzt und im vorhergehenden Status noch \textit{offline} war. Falls dann im Vorfeld Daten angelegt wurden, die nur lokal gespeichert werden konnten, werden diese im Falle des Synchronisierens ausgelesen, auf dem Server gespeichert und in der lokalen Datenbank wieder als "zum Server synchronisiert" gekennzeichnet.\\
In diesem 2. Meilenstein wurde der Programmcode refaktorisiert, um die Umsetzung weiterer Funktionen zu erleichtern. Im Folgenden ist es nunmehr nötig die Daten, die während der Zeit im \textit{Offline}-Modus angelegt wurden, in der Methode \textit{checkSync()} einzutragen.\\
Unter der vorherigen Architektur hätte die Logik in jede Verbidnungs-Operation kopiert werden müssen und hätte somit zu einer großen \textit{Code}-Redundanz geführt. Diese Redundanz sollte unbedingt umgangen werden und deshalb ist diese zentralisierte Stelle zum Überprüfen der zu übetragenden Daten umgesetzt worden.

\section{Anpassungen der Oberfläche}
\label{sec:anpassungen-oberflaeche}
Die Oberfläche wurde soweit angepasst, dass ein durchgängige \textit{Corporate Design} erkennbar ist. Die Oberfläche, insbesondere \textit{Buttons} und Dialoge wurden in diesem Schritt angepasst.\footcite{Android-Oberflaechen}\\
Dabei wurden zwei \textit{Designs}, jeweils für den Login- und den Registrieren-\textit{Button}, umgesetzt. Dazu war es nötig eine \ac{XML}-Datei anzulegen, die dann die Eigenschaften der Schaltfläche zu beschreiben hat (Quellcode \ref{lst:BtnSignInNotClicked}):
\lstinputlisting[caption=Design des Login-\textit{Buttons}, label=lst:BtnSignInNotClicked, style=xml]{content/listings/ButtonSignInStyle.xml}
Im Anhang (siehe \ref{sec:UserStory}) kann man die \textit{UserStory} in der nativen App in Gänze einsehen.\\ 
Weiterhin wurden die Dialogfenster mit Animationen versehen, um eine zum Betriebssystem passende \textit{Usability} gewährleisten zu können. Dazu musste ähnlich zum \textit{Button} eine \ac{XML}-Datei angelegt werden und darin dann die Bewegungen des Fensters beschrieben werden (Quellcode \ref{lst:slideRight}).
\lstinputlisting[caption=Dialog-Animation, label=lst:slideRight, style=xml]{content/listings/slide_right.xml}
Weiterhin wurden die \textit{Icons} zur Kenntlichmachung des Verbindungsstatus gegen zum Design passende ausgetauscht und ein App-\textit{Icon} eingefügt.

\section{Implementierung der Statistik}
\label{sec:implementierung-statistik}
Um Fortschritte des Nutzers anzeigen zu können, wurde eine Übersicht mit den eingetragenen Trainingsleistungen implementiert. Diese ist für jede Übung über einen längeren Klick auf das Übungs-Feld erreichbar. In dem Balkendiagramm ist das Produkt aus dem Trainingsgewicht, der Wiederholungszahl und der Satzzahl dargestellt.\\
Zur Darstellung wird die \textit{Xamarin-Extension BarChart} verwendet. Das Paket wird in das Projekt eingebunden und kann durch einen einfachen Aufruf mit der Übergabe eines Daten-\textit{Arrays} verwendet werden (Quellcode: \ref{lst:statistik}).\\
Zur Generierung der Diagramm-Daten waren Informationen über die Übung, den Trainingsplan und den \textit{User} nötig. Diese Daten werden über einen \textit{Intent} an die \textit{StatisticActivity} übergeben, in welcher dann die benötigten Daten abgerufen werden.
\lstinputlisting[caption=Statistik \textit{Activity}, label=lst:statistik, style=sharpc]{content/listings/StatisticActivity.cs}

\section{Fazit aus Meilenstein 2}
\label{sec:fazit-meilenstein-2}
Zusammenfassend kann festgehalten werden, dass der zweite Meilenstein zur Optimierung der nativen Adnroid-Applikation beigetragen hat. So konnte durch das Auslagern der Synchronisation Last vom dem \textit{UIThread} genommen werden. Die daraus folgenden Vorteile wurden bereits in Kapitel \ref{cha:native-app} vorgestellt.\\
Das es sich am Ende der Entwicklung um einen Messeprototypen handeln soll, ist das Design von besonderem Interesse. Diese Anforderung wurde im ersten Meilenstein zurückgestellt, um zuallererst einen Fokus auf den technischen Vergleich legen zu können. Nach der Sondierung der besseren Möglichkeit sollte diese dann weiter ausgebaut werden. Deshalb hat diese App eine Verbesserung der Oberflächen erhalten.\\
Die Statistik (ein Kann-Kriterium aus \ref{sec:Pflichtenheft}) wurde zusätzlich umgesetzt, da das Interesse an der Umsetzung eines langen Klicks und der Möglichkeiten der \textit{BarChart-Extension} groß waren.\\
Abschließend kann festgehalten werden, dass der zweite Meilenstein der nativen Applikation die letzten Verfeinerungen zum Prototypen gegeben hat.\\
Der Prototyp besitzt nun eine \textit{UserStory}, die alle benötigten Funktionen für die Entwicklung einer vollends ausgereiften App beinhaltet. So kann die Funktionalität des Speicherns eines Trainings auf das Speichern von Übungen und Trainingsplänen übertragen werden. Zudem kann das Abrufen von weiteren Daten, zum Beispiel für die Umsetzung von Administrator-Funktionen, mit den bestehenden Möglichkeiten umgesetzt werden.

%UserStory
\begin{figure}[!htbp]
\centering
\includegraphics[width=\textwidth, angle={90}]{content/images/UserStory}
\caption{UserStory}
\label{pic:UserStory}
\end{figure}

In der Abbildung \ref{pic:UserStory} ist die Übersicht der Oberflächen zu erkennen (größere Abbildungen sind im Anhnag unter Abschnitt \ref{sec:UserStory} zu finden). Darüber hinaus sind auch die Verbindungen zwischen den Oberflächen eingezeichnet, um sehen zu können, wie der \textit{Workflow} verläuft.\\
Der Einstieg des Benutzers in die Android-App geschieht über die Startseite, die die Auswahl zwischen Login und Registrierung darstellt. Wählt der Benutzer die Registrierung, erscheint der entsprechende Dialog. Um diese Funktion nutzen zu können, muss das Gerät eine Verbindung zum Server besitzen. 
Wenn sich der Benutzer einloggen möchte, erscheint der Login-Dialog. Um sich offline anmelden zu können, muss der Login einmal im Online-Modus durchgeführt worden sein, damit die App die Benutzerinformationen komplett vom Server erhalten hat.\\
Nach einem erfolgreichen Login, sei es offline oder online, werden die Trainingspläne des Benutzers als Liste angezeigt. Durch einen Klick auf den entsprechenden Plan, öffnet sich eine neue Seite mit den zu diesem Plan zugewiesenen Übungen.\\
Auf der folgenden Übersichtsseite werden die Übungen mit Namen und Beschreibung aufgelistet und der Benutzer hat nun zwei Möglichkeiten, um zu interagieren. Zum einen kann man durch das Auswählen einer Übung zu der Trainingsseite gelangen. Auf dieser Seite kann der Benutzer dann die Daten für diese Übung zu einem Training eintragen. Benötigt werden dazu das Gewicht, die Anzahl der Sätze und die Wiederholungszahl.
Als zweite Möglichkeit kann durch einen langen Klick auf eine Übung eine Statistik zu den dazu eingetragenen Trainings aufgerufen werden. Diese Statistik wird als Balkendiagramm angezeigt und gibt einen Überblick über die Leistungen des Benutzers. Der Index berechnet sich als Produkt der eingetragenen Werte Gewicht, Satzzahl und Wiederholungen.\\
Um in der Statistik alle Daten angezeigt zu bekommen, muss diese im Online-Modus aufgerufen werden, um die Daten auf das Gerät laden zu können. Dann können offline Trainings eingetragen werden und diese erscheinen unverzüglich in der Statistik. Wird die Statistik im Offline-Modus zum ersten Mal aufgerufen, werden nur die offline angelegten Trainings angezeigt.