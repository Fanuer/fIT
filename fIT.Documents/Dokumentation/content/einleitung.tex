\chapter{Einleitung}
\label{cha:einleitung}
In diesem Kapitel wird das grundlegende Problem und die daraus resultierende Zielsetzung erläutert. Anschließend wird grob auf das Vorgehen zur Lösung dieser Ziele eingegangen.

\section{Problemstellung}
\label{sec:problemstellung}
Momentan besitzen 57\% der Deutschen ein Smartphone. Somit hat sich die Zahl der Smartphone-Nutzer seit Ende 2011 mehr als verdoppelt.\footcite{Statista-SmartphoneNutzung}\\
Durch die verstärkte Nutzung geraten \glspl{App} immer mehr in den Fokus. Applikationen haben sich im Laufe der Zeit im Alltag ausgebreitet und sind mittlerweile für den Endnutzer unverzichtbar geworden. Sei es beim Online-Shopping, Chatten oder der Navigation. Überall finden diese kleinen Programme ihre Verwendung.\\
Dabei ist es besonders wichtig, dass eine konstante Internetverbindung besteht, um den kompletten Funktionsumfang nutzen zu können. Bis die Umsetzung eines flächendeckenden freien WLANs in Deutschland abgeschlossen ist, benötigt man eine gute Verbindung über seinen Netzbetreiber. Diese ist aber noch nicht vollständig und ausreichend im ganzen Land verfügbar.\\
Deshalb ist es notwendig, dass Apps versuchen Verbindungsabbrüche für den Benutzer zu überbrücken. Dabei besteht die Möglichkeit einer kurzzeitigen Zwischenspeicherung von Daten, solange die Internetverbindung nicht bereitsteht. Änderungen, die in dieser Zeit gemacht werden, sollen aufgenommen und später zur Verfügung gestellt werden damit man auf all seinen Endgeräten einen einheitlichen Stand der Daten hat.\\
Zur Umsetzung dieser Anforderungen können verschiedene Möglichkeiten genutzt werden.
\section{Zielsetzung}
\label{sec:zielsetzung}
Das Hauptziel dieser Arbeit ist der Wissenserwerb der Projektdurchführenden im Bereich der verlässlichen mobilen Applikationen. \\
Hierzu sollen zwei verlässliche Applikationen unter Nutzung verschiedener Techniken entworfen, umgesetzten und getestet werden. Als Beispiel soll eine Applikation entwickelt werden, die es ermöglichen den Trainingsfortschritt beim Krafttraining darzustellen, aufzunehmen und dauerhaft zu speichern. Zum Speichern der Benutzerdaten, wie Trainingspläne, Übungen und Trainings, wird ein Server benötigt, der die Anfragen der mobilen Geräte annimmt und verarbeitet.\\
Bei den Applikationen wird während der Entwicklungsphase entschieden, welche der beiden Apps zu einem lauffähigen Messeprototypen weiterentwickelt wird. Diese Einschätzung geschieht aufgrund der Erkenntnisse, welche durch die verschiedenen Umsetzungen gewonnen wurden.\\
Der Prototyp soll es dem Benutzer ermöglichen durch seine Trainingspläne mit den zugehörigen Übungen zu navigieren und die Daten eines Trainings eingeben zu können. Zudem soll es möglich sein die letzten Trainingseinheiten einzusehen. Das soll unabhängig davon funktionieren, ob das Endgerät eine Verbindung zum Server aufbauen kann oder nicht.
\section{Vorgehensweise}
\label{sec:vorgehensweise}
Zum Erreichen der Ziele muss als Erstes eine genauere Betrachtung der entstandenen Problematik durchgeführt werden. Anschließend werden benötigte Kenntnisse für das Erreichen der Ziele gesammelt. Diese gehen in die Planung der allgemeinen Architektur ein, welche Grundlage der späteren Implementierungen für Server und Clients ist. Nachdem die Architektur für Server und Clients definiert wurde, können diese umgesetzt werden. Die gewonnen Erkenntnisse aus den Implementierungen werden anschließend gegenübergestellt. Dies bildet die Grundlage für die Entscheidung, welche der beiden zu einem Messeprototypen weiterentwickelt wird. Abschließend wird die Erweiterung zum Messe-Prototypen umgesetzt.\\
Projektbegleitend wird eine Dokumentation erstellt, welche jeweils die durchgeführten Maßnahmen und gewonnen Kenntnisse widerspiegelt. 
\section{Arbeitsaufteilung}
\label{sec:Arbeitsaufteilung}
Da diese Arbeit von zwei Personen durchgeführt wird, muss an dieser Stelle noch aufgeschlüsselt werden, welche Aufgaben von wem durchgeführt werden. Hierzu dient die nachfolgende Tabelle \ref{tbl:arbeitsaufteilung}. Zur Wahrung der größtmöglichen Transparenz wurde zwischen der Dokumentation und der Implementierung der benötigten Komponenten unterschiedenen. 

\begin{table}[]
\caption{Arbeitsaufteilung}
\label{tbl:arbeitsaufteilung}
\begin{tabular}{|l|c|c|}
\hline
{\bf Aufgaben}                                                    & \multicolumn{2}{c|}{realisiert von}                                    \\
                                                                  & \multicolumn{1}{l|}{Kevin Schie} & \multicolumn{1}{l|}{Stefan Suermann} \\

{\bf Implementierung}                                             &                                &                                     \\
\hline
Server                                                            &                                 & X                                   \\
\hline
native App                                                        & X                               &                                     \\
\hline
Webapp                                                            &                                 & X                                   \\
\hline
Messeprototyp                                                     & X                               &                                     \\
\hline
                                                                  &                                 &                                     \\
{\bf Dokumentation}                                               &                                 &                                     \\
\hline
Einleitung                                                        &                                 & X                                    \\
\hline
Problemanalyse                                                    & X                               &                                     \\
\hline
Grundlagen                                                        & X                               &                                     \\
\hline
Architektur                                                       &                                 & X                                   \\
\hline
Aspekte der Realisierung                                          &                                 & X                                   \\
\hline
\makecell[l]{Realisierung der serverseitigen\\Implementierung}    &                                 & X                                   \\
\hline
\makecell[l]{Relisierung der clientseitigen\\Implementierung als native App}     & X                               &                                     \\
\hline
\makecell[l]{Relisierung der clientseitigen\\Implementierung als Webapplikation} &                                 & X                                   \\
\hline
\makecell[l]{Gegen{\"u}berstellung der\\clientseitigen Implementierung}        & X                               & X                                   \\
\hline
\makecell[l]{Weiterentwicklung eines\\Clients zu einem Messeprototyp}         & X                               &                                     \\
\hline
Fazit                                                             &                                 & X                                   \\
\hline
Pflichtenheft                                                     &                                 & X                                   \\
\hline
Cache Post                                                        & X                               &                                     \\
\hline
User-Story in der nativen App                                     & X                               &                                     \\
\hline
Funktionsumfang                                                   & X                               &   \\
\hline
                                 
\end{tabular}
\end{table}

%\begin{itemize}
%	\item Wie wird vorgegangen, um das Ziel zu erreichen?
%	\item Warum ist die Arbeit so gegliedert, wie sie gegliedert ist?
%	\item Welche Aspekte werden nicht behandelt \textbf{und} warum?
%\end{itemize}

%{TODO: Rausnehmen}
%Beschreibung: Wer was gemacht hat -> Übersicht je Kapitel
%
%Mobile Applikationen sind im täglichen Leben allgegenwärtig.
% 
%Eine Herausforderung bei diesen Anwendungen ist es, dass sie verlässlich funktionieren müssen, da ansonsten %ein Schaden auftritt, welcher sogar lebensbedrohlich- oder zumindest finanziell sein kann.
%Da dieses Problem in unterschiedlichen Anwendungen  immer wieder auftaucht, ist es sinnvoll, hierfür einen %generischen Ansatz anzubieten.
%
%Für mobile Endgeräte können zwei unterschiedliche Lösungsansätze verfolgt werden: 
%\begin{itemize}
%\item die Entwicklung nativer Apps und
%\item die Entwicklung mobiler Webseiten.
%\end{itemize}
%Diese beiden Lösungsansätze sollen unter dem Aspekt der Verlässlichkeit gegenübergestellt werden.
%
%Im Zuge der Arbeit soll geprüft werden, ob Caching besser über eine native App oder über eine responsive %Web-Applikation umgesetzt werden können. 
%Hierbei sollen die Vor- und Nachteile von mobilen Webapplikationen gegenüber nativen Apps beleuchtet werden.
%
%
%In diesem Unterkapitel sollten folgende Punkte behandelt werden:
%\begin{itemize}
%	\item Was ist das Problem?
%	\item Problemgeschichte?
%\end{itemize}
