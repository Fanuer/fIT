Übersicht wer was gemacht hat

\chapter{Einleitung}
\label{cha:einleitung}
In diesem Kapitel wird das grundlegende Problem und die daraus resultierende Aufgabenstellung erläutert.

\section{Problemstellung}
\label{sec:problemstellung}

\textbf
Momentan besitzen 57\% der Deutschen ein Smartphone. Somit hat sich die Zahl der Smartphone-Nutzer seit Ende 2011 mehr als verdoppelt.\footcite{Statista-SmartphoneNutzung}
Durch die verstärkte Nutzung, geraten Applikationen (Apps) - kleine Programme für mobile Endgeräte - immer mehr in den Fokus.
Apps haben sich im Laufe der Zeit im Alltag breit gemacht und sind mittlerweile für den Endnutzer unverzichtbar geworden. Sei es beim Online-Shopping, Chatten oder der Navigation. Überall finden Applikationen ihre Verwendung.
Dabei ist es besonders wichtig, dass eine konstante Internetverbindung besteht, um den kompletten Funktionsumfang nutzen zu können.
Bis die Umsetzung eines flächendeckenden freien WLANs in Deutschland abgeschlossen ist, benötigt man eine gute Verbindung über seinen Netzbetreiber. Diese ist aber nicht vollständig und ausreichend im ganzen Land verfügbar.

Auf Grund dessen ist es notwendig, dass die Applikationen versuchen Verbindungsabbrüche für den Benutzer zu überbrücken. Dabei besteht die Möglichkeit einer kurzzeitigen Zwischenspeicherung von Daten, die vom Benutzer eingesehen oder verwendet werden können, solange die Internetverbindung nicht bereitsteht. Änderungen, die in dieser Zeit gemacht wurden, sollen auch aufgenommen und später zur Verfügung gestellt werden.\\
Zur Umsetzung dieser Idee bestehen zwei Möglichkeiten. Zum einen kann eine mobile Web- oder eine native Applikation genutzt werden. [Zitat eines Gurus]




\section{Zielsetzung}
\label{sec:zielsetzung}
Das Ziel dieser Arbeit soll es sein, die Architektur für eine verlässliche Applikation zu entwerfen. Zum einen wird die Verarbeitung und Umsetzung auf einem Windows-Server erläutert. Auf der anderen Seite werden parallel zwei Applikationen zum verlässlichen Zugriff entwickelt und anhand dessen beleuchtet, welche Umsetzung für den angegeben Sachverhalt angemessener erscheint. Für die Umsetzung der Webapplikation wird das ASP.Net-Framework verwendet. Die native Applikation wird aus technischen Gründen mit Hilfe von Xamarin für Android entwickelt. Die Auswahl des Android-Betriebssystems besteht darin, dass Tests auch ohne Komplikationen oder Beschränkungen des Hersteller auf eigenen Geräten problemlos durchgeführt werden können.\\
Die vorteilhaftere Möglichkeit wird zu einem Prototypen mit rudimentären Funktionen und Design weiterentwickelt. Dabei besteht dann die Möglichkeit einen Trainingsplan zu erstellen und auf die Trainingsdaten der letzten fünf Trainings - unabhängig von der Internetverbidnung - zuzugreifen.

%\begin{itemize}
%	\item Was soll mit der Arbeit erreicht werden? Welche Ziele werden angestrebt? Möglichst kurz und präzise geplante Ergebnisse umreißen. $/rightarrow$ Daran werden Ihre Resultate am Ende gemessen!
%\end{itemize}


\section{Vorgehensweise}
\label{sec:vorgehensweise}
Nachdem nun die Notwendigkeit von verlässlichen Applikationen und das Ziel der Arbeit definiert wurden, befasst sich das folgende Kapitel~\ref{cha:problemanalyse} mit der Problemanalyse im Hinblick auf die Umsetzung mit den beiden herangezogenen Varianten nativer- und Webapplikation.


%\begin{itemize}
%	\item Wie wird vorgegangen, um das Ziel zu erreichen?
%	\item Warum ist die Arbeit so gegliedert, wie sie gegliedert ist?
%	\item Welche Aspekte werden nicht behandelt \textbf{und} warum?
%\end{itemize}







%{TODO: Rausnehmen}
%Beschreibung: Wer was gemacht hat -> Übersicht je Kapitel
%
%Mobile Applikationen sind im täglichen Leben allgegenwärtig.
% 
%Eine Herausforderung bei diesen Anwendungen ist es, dass sie verlässlich funktionieren müssen, da ansonsten %ein Schaden auftritt, welcher sogar lebensbedrohlich- oder zumindest finanziell sein kann.
%Da dieses Problem in unterschiedlichen Anwendungen  immer wieder auftaucht, ist es sinnvoll, hierfür einen %generischen Ansatz anzubieten.
%
%Für mobile Endgeräte können zwei unterschiedliche Lösungsansätze verfolgt werden: 
%\begin{itemize}
%\item die Entwicklung nativer Apps und
%\item die Entwicklung mobiler Webseiten.
%\end{itemize}
%Diese beiden Lösungsansätze sollen unter dem Aspekt der Verlässlichkeit gegenübergestellt werden.
%
%Im Zuge der Arbeit soll geprüft werden, ob Caching besser über eine native App oder über eine responsive %Web-Applikation umgesetzt werden können. 
%Hierbei sollen die Vor- und Nachteile von mobilen Webapplikationen gegenüber nativen Apps beleuchtet werden.
%
%
%In diesem Unterkapitel sollten folgende Punkte behandelt werden:
%\begin{itemize}
%	\item Was ist das Problem?
%	\item Problemgeschichte?
%\end{itemize}
