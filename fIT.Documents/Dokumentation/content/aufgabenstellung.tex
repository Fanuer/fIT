\chapter*{Aufgabenstellung}
\addcontentsline{toc}{chapter}{Aufgabenstellung}  
\label{cha:augabenstellung}

Mobile Applikationen sind im täglichen Leben allgegenwärtig.
 
Eine Herausforderung bei diesen Anwendungen ist es, dass sie verlässlich funktionieren müssen, da ansonsten ein Schaden auftreten kann. Da dieses Problem in unterschiedlichen Anwendungen  immer wieder auftaucht, ist es sinnvoll, hierfür einen generischen Ansatz anzubieten. 

Für mobile Endgeräte können zwei unterschiedliche Lösungsansätze verfolgt werden: 
\begin{itemize}
\item die Entwicklung nativer Apps und
\item die Entwicklung mobiler Webseiten.
\end{itemize}
Diese beiden Lösungsansätze sollen unter dem Aspekt der Verlässlichkeit gegenübergestellt und verglichen werden.

Der aus der Evaluation hervorgegangene günstigere Lösungsweg soll in einem konkreten Messeprototypen implementiert werden.

Als Beispiel soll eine Applikation für mobile Endgeräte erstellt werden, in der ein Nutzer die Fortschritte seines Trainings festhalten kann. 
Die dabei entstandenen Daten sollen zentral auf einem Server verwaltet werden. 
Dieses Szenario ist zwar kein klassisches Beispiel für eine verlässliche Anwendung, allerdings lassen sich an diesem Beispiel alle Konzepte aufzeigen.
