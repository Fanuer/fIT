%% Dokumentenklasse (Koma Script)
\documentclass[%
   %draft,     % Entwurfsstadium
   final,      % fertiges Dokument
%%%% --- Schriftgröße ---
   11pt,
%%%% --- Sprache ---
   english,
   ngerman,           % Letzte Sprache ist die Hauptsprache, andere muss erst ausgewählt werden.
%%%% === Seitengröße ===
   a4paper,
%%%% === Optionen für den Satzspiegel ===
   %BCOR5mm,         % Zusaetzlicher Rand auf der Innenseite (Bindekorrektur)
   %DIV11,           % Seitengroesse (siehe Koma Skript Dokumentation!)
   %DIVcalc,         % automatische Berechnung einer guten Zeilenlaenge
   1.1headlines,     % Zeilenanzahl der Kopfzeilen
   %headinclude,     % Kopf einbeziehen
   headinclude=false,      % Kopf nicht einbeziehen
   %footinclude,     % Fuss einbeziehen
   footinclude=false,      % Fuss nicht einbeziehen
   %mpinclude,       % Margin einbeziehen
   mpinclude=false,        % Margin nicht einbeziehen
   pagesize,         % Schreibt die Papiergroesse in die Datei.
                     % Wichtig fuer Konvertierungen
%%%% === Layout ===
   %oneside,          % einseitiges Layout
   twoside,         % Seitenraender für zweiseitiges Layout
   onecolumn,        % Einspaltig
   %twocolumn,       % Zweispaltig
   %openany,         % Kapitel beginnen auf jeder Seite
   %openright,        % Kapitel beginnen immer auf der rechten Seite
                     % (macht nur bei 'twoside' Sinn)
   %cleardoubleplain,    % leere, linke Seite mit Seitenstil 'plain'
   %cleardoubleempty,    % leere, linke Seite mit Seitenstil 'empty'
   titlepage,        % Titel als einzelne Seite ('titlepage' Umgebung)
   %notitlepage,     % Titel in Seite integriert
%%%% --- Absatzeinzug ---
   %                 % Absatzabstand: Einzeilig,
   %parskip,         % Freiraum in letzter Zeile: 1em
   %parskip*,        % Freiraum in letzter Zeile: Viertel einer Zeile
   %parskip+,        % Freiraum in letzter Zeile: Drittel einer Zeile
   %parskip-,        % Freiraum in letzter Zeile: keine Vorkehrungen
   %                 % Absatzabstand: Halbzeilig
   %halfparskip,     % Freiraum in letzter Zeile: 1em
   %halfparskip*,    % Freiraum in letzter Zeile: Viertel einer Zeile
   %halfparskip+,    % Freiraum in letzter Zeile: Drittel einer Zeile
   parskip=half,      % Freiraum in letzter Zeile: keine Vorkehrungen
   %                 % Absatzabstand: keiner
   %parindent,       % Eingerückt (Standard)
%%%% --- Kolumnentitel ---
   headsepline,      % Linie unter Kolumnentitel
   %headnosepline,   % keine Linie unter Kolumnentitel
   %footsepline,     % Linie unter Fussnote
   %footnosepline,   % keine Linie unter Fussnote
%%%% --- Kapitel ---
   %chapterprefix,   % Ausgabe von 'Kapitel:'
   chapterprefix=false,
   version=first,  % keine Ausgabe von 'Kapitel:'
%%%% === Verzeichnisse (TOC, LOF, LOT, BIB) ===
   listof=totoc,      % Tabellen & Abbildungsverzeichnis ins TOC
   %idxtotoc,        % Index ins TOC
   bibliography=totoc,
   verson=first,         % Bibliographie ins TOC
   %bibtotocnumbered, % Bibliographie im TOC nummeriert
   %liststotocnumbered, % Alle Verzeichnisse im TOC nummeriert
   toc=graduated,        % eingereuckte Gliederung
   %tocleft,         % Tabellenartige TOC
   %listsindent,      % eingereuckte LOT, LOF
   %listsleft,       % Tabellenartige LOT, LOF
   %pointednumbers,  % Überschriftnummerierung mit Punkt, siehe DUDEN !
   %pointlessnumbers, % Überschriftnummerierung ohne Punkt, siehe DUDEN !
   %openbib,         % alternative Formatierung des Literaturverzeichnisses
%%%% === Matheformeln ===
   %leqno,           % Formelnummern links
   fleqn,            % Formeln werden linksbuendig angezeigt
]{scrbook}%     Klassen: scrartcl, scrreprt, scrbook
% -------------------------------------------------------------------------

% -------------- Daten für die Titelseite --> diese müssen angepasst werden!
\newcommand*{\thedockind}{Bachelorthesis}
\newcommand*{\thetitle}{Verl{\"a}ssliche mobile Anwendungen}
\newcommand*{\thesubtitle}{Untersuchungen am Beispiel einer Fitness-App}
\newcommand*{\theauthor}{Kevin Schie / Stefan Suermann}
\newcommand*{\thematriculationnumber}{2012013 / 2012027} % Matrikelnummer
\newcommand*{\thebirthday}{04.07.1993 / 13.12.1987} % Geburtstag
\newcommand*{\thedegree}{Bachelor of Science}
\newcommand*{\themajor}{Praktische Informatik} % Studiengang
\newcommand*{\thedate}{\today} % \today kann durch ein Datum erstetzt werden.
\newcommand*{\thebetreuer}{Prof. Dr. Johannes Ecke-Sch{\"u}th}
\newcommand*{\thezweitbetreuer}{Prof. Dr. Klaus-Dieter Kr{\"a}geloh}
% -------------- Ende Daten für die Titelseite


%%% Doc: ftp://tug.ctan.org/pub/tex-archive/macros/latex/required/babel/babel.pdf
% Languagesetting
\usepackage{babel}	% Sprache

\usepackage{textpos} 


\usepackage{fixltx2e}	% Verbessert einige Kernkompetenzen von LaTeX2e
\usepackage{ellipsis}	% Korrigiert den Wei�raum um Auslassungspunkte

\usepackage{placeins} 

\usepackage{ifpdf}
\ifpdf
\pdfinfo {
	/Author (\theauthor)
	/Title (\thetitle)
	/Subject ()
	/Keywords ()
%	/CreationDate (D:YYYYMMTTHHMMSS)
}
\fi



%%% Doc: www.cs.brown.edu/system/software/latex/doc/calc.pdf
% Calculation with LaTeX
\usepackage{calc}

%%% Doc: ftp://tug.ctan.org/pub/tex-archive/macros/latex/contrib/xcolor/xcolor.pdf
% Farben
% Incompatible: Do not load when using pstricks !
\usepackage[
	table % Load for using rowcolors command in tables
]{xcolor}

\usepackage{tikz}
\usetikzlibrary{% 
   arrows,% 
   calc,% 
   fit,% 
   patterns,% 
   plotmarks,% 
   shapes.geometric,% 
   shapes.misc,% 
   shapes.symbols,% 
   shapes.arrows,% 
   shapes.callouts,% 
   shapes.multipart,% 
   shapes.gates.logic.US,% 
   shapes.gates.logic.IEC,% 
   er,% 
   automata,% 
   backgrounds,% 
   chains,% 
   topaths,% 
   trees,% 
   petri,% 
   mindmap,% 
   matrix,% 
   calendar,% 
   folding,% 
   fadings,% 
   through,% 
   positioning,% 
   scopes,% 
   decorations.fractals,% 
   decorations.shapes,% 
   decorations.text,% 
   decorations.pathmorphing,% 
   decorations.pathreplacing,% 
   decorations.footprints,% 
   decorations.markings,% 
   shadows} 
   
%%% Doc: ftp://tug.ctan.org/pub/tex-archive/macros/latex/required/graphics/grfguide.pdf
% Bilder
\usepackage[%
	%final,
	%draft % do not include images (faster)
]{graphicx}


% bessere Abstaende innerhalb der Tabelle (Layout))
% -------------------------------------------------
%%% Doc: ftp://tug.ctan.org/pub/tex-archive/macros/latex/contrib/booktabs/booktabs.pdf
\usepackage{booktabs}


%%% Doc: ftp://tug.ctan.org/pub/tex-archive/macros/latex/contrib/enumitem/enumitem.pdf
% Better than 'paralist' and 'enumerate' because it uses a keyvalue interface!
% Do not load together with enumerate.
%\usepackage{enumitem}

\usepackage{paralist}


%%% Doc: http://www.ctan.org/tex-archive/macros/latex/contrib/acronym/acronym.pdf
% Usage:
%        Definition: \acro{ acronym }[ short name ]{ full name }
%        Nutzung im Text: \ac{acronym}
 \usepackage[
 	footnote,	% Full names appear in the footnote
 	%smaller,		% Print acronym in smaller fontsize
 	%printonlyused %
 ]{acronym}
%\chapter*{Abkürzungsverzeichnis}
\begin{acronym}[BiPRO ] %Längster Begriff
\setlength{\itemsep}{-\parsep}
	\acro{ACL}{Access Control Lists}
	\acro{AES}{Advanced Encryption Standard}
	\acro{ANR}{Application Not Responding}
	\acro{API}{Application Programming Interface}
	\acro{CORS}{Cross-origin resource sharing}
	\acro{CRUD}{Create Read Update Delete}
	\acro{DRAM}{Dynamischer RAM}
	\acro{HTTP}{Hyper Text Transfer Protocol}
	\acro{OR-Mapper}{objekt-relationaler Mapper}
	\acro{PCL}{Portable Class Library}
	\acro{RAM}{Random Access Memory}
	\acro{SRAM}{Statischer RAM}
	\acro{TCP}{Transmission Control Protocol}
	\acro{URI}{Uniform Resource Identifier}
	\acro{VM}{Virtuelle Maschine}
	\acro{XML}{Extensible Markup Language}
	\acro{.apk}{Android Package}
	%usw.
\end{acronym}



%% Kopf und Fusszeilen====================================================
%%% Doc: ftp://tug.ctan.org/pub/tex-archive/macros/latex/contrib/koma-script/scrguide.pdf
\usepackage[%
   automark,         % automatische Aktualisierung der Kolumnentitel
   %nouppercase,      % Grossbuchstaben verhindern
   %markuppercase    % Grossbuchstaben erzwingen
   %markusedcase     % vordefinierten Stil beibehalten
   %komastyle,       % Stil von Koma Script
   %standardstyle,   % Stil der Standardklassen
]{scrpage2}



%% UeberSchriften (Chapter und Sections) =================================
% -- Ueberschriften komlett Umdefinieren --
%%% Doc: ftp://tug.ctan.org/pub/tex-archive/macros/latex/contrib/titlesec/titlesec.pdf
\usepackage{titlesec}

% -- Section Aussehen veraendern --
% --------------------------------
%% -> Section mit Unterstrich
% \titleformat{\section}
%   [hang]%[frame]display
%   {\usekomafont{sectioning}\Large}
%  {\thesection}
%   {6pt}
%   {}
%   [\titlerule \vspace{0.5\baselineskip}]
% --------------------------------

% -- Chapter Aussehen veraendern --
% --------------------------------
\titleformat{\chapter}[block]	% {command}[shape]
  {\usekomafont{chapter}\huge\sffamily\bfseries}	% format
  {   										% label
  {\thechapter.} \filright%
  }%}
  {1pt}										% sep (from chapternumber)
  {\vspace{0.5pc} \filright}   % {before}[after] (before chaptertitle and after)
  [\vspace{0.5pc} \filright {}]

% \titleformat{\chapter}[]%
%    {\usekomafont{chapter}\huge\sffamily\bfseries}%
%    {\thechapter}%
%    {1em}%
%    {}%


\usepackage{rotating}


\usepackage[numbers,square]{natbib}
%\usepackage{cite}
%\bibliographystyle{dinat}
%\citestyle{alpha}
\bibliographystyle{alpha}


% Quotes =================================================================
%% Doc: ftp://tug.ctan.org/pub/tex-archive/macros/latex/contrib/csquotes/csquotes.pdf
% Advanced features for clever quotations
\usepackage[%
   babel,            % the style of all quotation marks will be adapted
                     % to the document language as chosen by 'babel'
   german=quotes,		% Styles of quotes in each language
   %german=guillemets,
   english=british,
   french=guillemets
]{csquotes}
\usepackage{floatflt}

\usepackage{wrapfig}

%\usepackage{subfigure}

\usepackage{blindtext}

\usepackage{listings}
\definecolor{lila}{RGB}{112, 6, 147}
\definecolor{kommentgreen}{RGB}{5,132,71}
\definecolor{grey}{RGB}{242,242,242}  
\definecolor{darkgreen}{named}{green}
\definecolor{darkblue}{named}{blue}
\definecolor{lightblue}{RGB} {63,95,191}
\definecolor{darkred}{named}{red}
\definecolor{grau}{named}{gray}
\definecolor{fh_orange}{rgb}{0.953,0.201,0}
\definecolor{fh_grau}{rgb}{0.76,0.75,0.76}

\definecolor{listinggray}{gray}{0.9}
\definecolor{lbcolor}{rgb}{0.9,0.9,0.9}

\lstset{
	tabsize=3,
	float=tbph,
	frame=single,
	extendedchars,
	breaklines=true,
	basicstyle=\fontsize{9pt}{9pt}\selectfont,
	columns=flexible, %ist notwendig, damit man Quelltext aus den Listings kopieren kann
	numbers=left, 
	numberstyle=\color{black},
	captionpos=b,
	aboveskip=7mm,
	backgroundcolor=\color{grey}
}

\lstdefinestyle{java}
{
	language=Java,
	keywordstyle=\color{lila},  	% underlined bold black keywords 
	identifierstyle=\color{blue}, 
	commentstyle=\color{kommentgreen}, % white comments 
	stringstyle=\color{black},
}

\lstdefinestyle{xml}
{
	language=xml,
	basicstyle=\fontsize{9pt}{9pt}\selectfont\color{kommentgreen},
	keywordstyle=\color{lila},  	% underlined bold black keywords 
	%Hier k�nnen bei Bedarf noch weitere Keywords eingetragen werden
	keywords={name, value, version, encoding, id, type, xmlns:xsi, ref, namespace},
	identifierstyle=\color{black},  
	stringstyle=\color{blue},  
	commentstyle=\color{lightblue},
	morecomment=[s]{<!--}{-->},
	rulecolor=\color{black}
}

\usepackage{multicol}

\usepackage{nameref}

\usepackage{hyperref}
\hypersetup{breaklinks=true}
\hypersetup{colorlinks=true,linkcolor=black,urlcolor=black,citecolor=black}
%\hypersetup{frenchlinks}	% Use small caps instead of color for links
%\hypersetup{pdfpagemode=FullScreen}
%\hypersetup{pdfstartpage=3}
%\hypersetup{pdfstartview=Fit}


% Tabellen ueber mehere Seiten
% ----------------------------
%%% Doc: ftp://tug.ctan.org/pub/tex-archive/macros/latex/contrib/carlisle/ltxtable.pdf
% \usepackage{ltxtable} % Longtable + tabularx
                        % (multi-page tables) + (auto-sized columns in a fixed width table)
% -> nach hyperref laden
%\usepackage{ltxtable}
%\usepackage{longtable}
\usepackage{tabulary}


% Schusterjunge und Hurenkinder verhindern
\clubpenalty=1000
\widowpenalty=1000
\displaywidowpenalty=1000

% Trennen von Bindestrich oder so ...
%\defaulthyphenchar=127

\usepackage[latin1]{inputenc}
\usepackage[T1]{fontenc}
\usepackage{alltt}
\usepackage{marvosym}
\usepackage{fancybox}
\usepackage[hang,small,bf]{caption}
\usepackage{float} 
\usepackage{multirow}
\addtokomafont{caption}{\sffamily\small}
\setkomafont{captionlabel}{\sffamily\small}

\newcommand{\keyword}[1]{\textbf{#1}}
%\newcommand{\filename}[1]{\texttt{#1}}
\newcommand{\inlinecode}[1]{\lstinline!#1!}



% Umbenennung von "Listings"
\addto\captionsngerman{%
  \renewcommand{\lstlistlistingname}{Quelltextverzeichnis}%
  \renewcommand{\lstlistingname}{Quelltext}%
  \renewcommand{\}}{}
}

\usepackage{blindtext}
\usepackage{helvet}	%Paket für die Schriftart

\renewcommand{\familydefault}{\sfdefault}	%sorgt für einheitliche Schriftart im gesamten Dokument
\setcounter{secnumdepth}{5} %legt die Ebene fest, bis zu der nummeriert wird
\setcounter{tocdepth}{5} %legt fest wieviele Ebenen im Inhaltsverzeichnis vorkommen


\begin{document}

	%\captionsetup[figure]{singlelinecheck=false} %sorgt für eine Linksbündige Bildunterschrift
	%\captionsetup[lstlisting]{singlelinecheck=off}
	\captionsetup{singlelinecheck=off}
	
  \sffamily		% Schriftart Serifenlos wählen
  \linespread {1.25}\selectfont %Zeilenabstand: 1.25 da er von Haus aus 1.2 ist und 1,25 * 1,2 = 1,5

	%Titelseite einfügen
	
\begin{titlepage}
		
%%%%%%%%%%%%%%%%%%%%%%%%%%%%%% -*- Mode: Latex -*- %%%%%%%%%%%%%%%%%%%%%%%%%%%%
%% 
%% pa_ba_titelblatt.tex 
%% 
%% Copyright (C) 2008 Alexander Sprack / Claudia Holz
%% 
%%%%%%%%%%%%%%%%%%%%%%%%%%%%%%%%%%%%%%%%%%%%%%%%%%%%%%%%%%%%%%%%%%%%%%%%%%%%%%%
  \begin{textblock}{6.5}(-1,-3)
    \begin{color}{fh_grau}
      \rule{7cm}{33cm}    
    \end{color}
  \end{textblock}
  \begin{textblock}{6.5}(-1.2,-0.7)
%  \includegraphics[width=3.8cm]{my-fh-logo}% selbst basteln, falls gewünscht! 
                                            % Das offizielle Logo ist nicht
                                            % gestattet!! Bitte BEACHTEN!!!
  \end{textblock}
  \begin{textblock}{6.5}(-0.8,1)
    {\Large \textsf{\thedockind}}            
  \end{textblock}

  \begin{textblock}{7}(4.5,2)
    {\noindent \huge 
      \textsf{\textbf{\thetitle\\[0.3cm] 
          \Large  \thesubtitle\\[0.05cm]
          }} }
  \end{textblock}


  \begin{textblock}{6}(4.5,6.5)\noindent
    \textsf{Am IT-Center Dortmund GmbH\\
    Studiengang \themajor \\
    erstellte \thedockind \\
    zur Erlangung des akademischen Grades\\
    \thedegree}
  \end{textblock}

  \begin{textblock}{6.5}(-0.4,10.0)
    \noindent
    \textsf{von \\
      \theauthor \\
      geb.\ am \thebirthday  \\
      Matr.-Nr. \thematriculationnumber\\[0.7cm]
      Betreuer:\\
       \noindent\hspace*{6mm} \thebetreuer \\
       \noindent\hspace*{6mm} \thezweitbetreuer\\ [0.5cm]
      Dortmund, \today}    
  \end{textblock}
	

\end{titlepage}



%\thispagestyle{empty}

	
	\frontmatter	 		%römische Nummerierung für Inhaltsverzeichnis aktivieren
	\tableofcontents	%Inhaltsverzeichnis erstellen
	\mainmatter	 			%Arabische Seitenummerierung

	\pagestyle{scrheadings}
	
	
	% ----------------- Einfügen des eigentlichen Textes
	
	\chapter*{Aufgabenstellung}
\addcontentsline{toc}{chapter}{Aufgabenstellung}  
\label{cha:augabenstellung}

Mobile Applikationen sind im täglichen Leben allgegenwärtig.
 
Eine Herausforderung bei diesen Anwendungen ist es, dass sie verlässlich funktionieren müssen, da ansonsten ein Schaden auftreten kann. Da dieses Problem in unterschiedlichen Anwendungen  immer wieder auftaucht, ist es sinnvoll, hierfür einen generischen Ansatz anzubieten. 

Für mobile Endgeräte können zwei unterschiedliche Lösungsansätze verfolgt werden: 
\begin{itemize}
\item die Entwicklung nativer Apps und
\item die Entwicklung mobiler Webseiten.
\end{itemize}
Diese beiden Lösungsansätze sollen unter dem Aspekt der Verlässlichkeit gegenübergestellt und verglichen werden.

Der aus der Evaluation hervorgegangene günstigere Lösungsweg soll in einem konkreten Messeprototypen implementiert werden.

Als Beispiel soll eine Applikation für mobile Endgeräte erstellt werden, in der ein Nutzer die Fortschritte seines Trainings festhalten kann. 
Die dabei entstandenen Daten sollen zentral auf einem Server verwaltet werden. 
Dieses Szenario ist zwar kein klassisches Beispiel für eine verlässliche Anwendung, allerdings lassen sich an diesem Beispiel alle Konzepte aufzeigen.
	
	Übersicht wer was gemacht hat

\chapter{Einleitung}
\label{cha:einleitung}
In diesem Kapitel wird das grundlegende Problem und die daraus resultierende Aufgabenstellung erläutert.

\section{Problemstellung}
\label{sec:problemstellung}
Momentan besitzen 57\% der Deutschen ein Smartphone. Somit hat sich die Zahl der Smartphone-Nutzer seit Ende 2011 mehr als verdoppelt.\footcite{Statista-SmartphoneNutzung}\\
Durch die verstärkte Nutzung geraten \glspl{App} immer mehr in den Fokus. Applikationen haben sich im Laufe der Zeit im Alltag ausgebreitet und sind mittlerweile für den Endnutzer unverzichtbar geworden. Sei es beim Online-Shopping, \textit{Chatten} oder der Navigation. Überall finden diese kleinen Programme ihre Verwendung.\\
Dabei ist es besonders wichtig, dass eine konstante Internetverbindung besteht, um den kompletten Funktionsumfang nutzen zu können. Bis die Umsetzung eines flächendeckenden freien WLANs in Deutschland abgeschlossen ist, benötigt man eine gute Verbindung über seinen Netzbetreiber. Diese ist aber noch nicht vollständig und ausreichend im ganzen Land verfügbar.\\
Deshalb ist es notwendig, dass die Apps versuchen Verbindungsabbrüche für den Benutzer zu überbrücken. Dabei besteht die Möglichkeit einer kurzzeitigen Zwischenspeicherung von Daten, die vom Benutzer eingesehen oder verwendet werden können, solange die Internetverbindung nicht bereitsteht. Änderungen, die in dieser Zeit gemacht werden, sollen auch aufgenommen und später zur Verfügung gestellt werden damit man auf all seinen Endgeräten einen einheitlichen Stand der Daten hat.\\
Zur Umsetzung dieser Anforderungen können verschiedene Möglichkeiten genutzt werden. Die beiden verbreitetsten Methoden sind native oder Web-Apps.
\section{Zielsetzung}
\label{sec:zielsetzung}
Ziel dieser Arbeit soll es sein, zwei unterschiedliche verlässliche Applikationen zu entwerfen, umzusetzen und im Anschluss zu testen. Diese sollen es ermöglichen den Trainingsfortschritt beim Krafttraining darzustellen, aufzunehmen und dauerhaft zu speichern.\\
Zum Speichern der Benutzerdaten, wie Trainingspläne, Übungen und Trainings, wird ein Server benötigt, der die Anfragen der mobilen Geräte annimmt und verarbeitet. Dafür soll ein Windows-Server implementiert (siehe Kapitel \ref{cha:server-impl}) und verwendet werden.\\
Bei den Applikationen wird während der Entwicklungsphase entschieden, welche der beiden Apps zu einem lauffähigen Messeprototypen weiterentwickelt wird. Diese Einschätzung kann jedoch erst getroffen werden, wenn verschiedene Umsetzungen in den einzelnen Applikationen getestet wurden.\\
Der Prototyp soll es dem Benutzer ermöglichen durch seine Trainingspläne mit den zugehörigen Übungen zu navigieren und die Daten eines Trainings eingeben zu können. Zudem soll es möglich sein die letzten Trainingseinheiten einzusehen. Das soll unabhängig davon funktionieren, ob das Smartphone eine Verbindung zum Server hat oder nicht.
\section{Vorgehensweise}
\label{sec:vorgehensweise}
Zum Erreichen der Ziele wird zuallererst ein Überblick über die Grundlagen der Umsetzung gegeben. Dabei werden dann schon die ersten Techniken vorgestellt, die für die Implementierung verwendet werden sollen. Folgend wird die allgemeine Architektur des Systems, bestehend aus den beiden Applikationen und dem Server, erläutert, um den Gesamtzusammenhang dieses Projektes in Gänze überblicken zu können. Darauf aufbauend wird jeweils detaillierter auf die Umsetzungen der Apps und des Servers, sowie die dabei verwendeten Technologien eingegangen. Daraufhin werden die Applikationen verglichen und entschieden, welche der beiden zu einem Messeprototypen weiterentwickelt wird. Abschließend wird die Erweiterung zum Prototypen vorgestellt und ein Rückblick auf das gesamte Projekt gegeben.



%\begin{itemize}
%	\item Wie wird vorgegangen, um das Ziel zu erreichen?
%	\item Warum ist die Arbeit so gegliedert, wie sie gegliedert ist?
%	\item Welche Aspekte werden nicht behandelt \textbf{und} warum?
%\end{itemize}







%{TODO: Rausnehmen}
%Beschreibung: Wer was gemacht hat -> Übersicht je Kapitel
%
%Mobile Applikationen sind im täglichen Leben allgegenwärtig.
% 
%Eine Herausforderung bei diesen Anwendungen ist es, dass sie verlässlich funktionieren müssen, da ansonsten %ein Schaden auftritt, welcher sogar lebensbedrohlich- oder zumindest finanziell sein kann.
%Da dieses Problem in unterschiedlichen Anwendungen  immer wieder auftaucht, ist es sinnvoll, hierfür einen %generischen Ansatz anzubieten.
%
%Für mobile Endgeräte können zwei unterschiedliche Lösungsansätze verfolgt werden: 
%\begin{itemize}
%\item die Entwicklung nativer Apps und
%\item die Entwicklung mobiler Webseiten.
%\end{itemize}
%Diese beiden Lösungsansätze sollen unter dem Aspekt der Verlässlichkeit gegenübergestellt werden.
%
%Im Zuge der Arbeit soll geprüft werden, ob Caching besser über eine native App oder über eine responsive %Web-Applikation umgesetzt werden können. 
%Hierbei sollen die Vor- und Nachteile von mobilen Webapplikationen gegenüber nativen Apps beleuchtet werden.
%
%
%In diesem Unterkapitel sollten folgende Punkte behandelt werden:
%\begin{itemize}
%	\item Was ist das Problem?
%	\item Problemgeschichte?
%\end{itemize}
	
	\chapter{Problemanalyse}
\label{cha:problemanalyse}
Im letzten Kapitel wurde die Aufgabenstellung grob beschrieben. Nun sollen die angesprochenen Probleme feiner analysiert werden, so dass sich die konkreten Ziele ergeben. Die Ziele bilden die Grundlage für die Entscheidungen zum weitere Vorgehen während der Umsetzung, welche in einem Soll-Konzept dargelegt wird.
\section{Problembeschreibung}
\label{sec:problembeschreibung}
Aus der groben Problembeschreibung lassen sich folgende technische Herausforderungen ablesen:\\
Der Server dient als zentrale Datenhaltung für verschiedene Clients. Darum ist es nötig, dass Client und Server dafür ausgelegt werden, über einen standardisierte Schnittstelle zu kommunizieren, um die anfallenden Daten auszutauschen. Diese Schnittstelle muss einen Authentifizierungs- und Autorisierungsmechanismus implementieren, so dass jeder Nutzer nur an seine Daten gelangt. Darüber hinaus ist die Schnittstelle so zu implementieren, dass diese Kommunikation (zumindest temporär) optional ist, sodass eine Ausfallsicherheit entsteht. Um diese zu gewährleisten, muss ein Client die Möglichkeit haben, die Verbindung zum Server zu prüfen. Schlägt diese Prüfung fehl, muss der Client mit entsprechenden Maßnahmen reagieren. Hierfür muss zum einen der Zugriff auf Funktionen, welche zwingend eine Verbindung benötigen, reguliert werden. Zum Anderen müssen lokal anfallende Daten bei fehlender Verbindung zwischengespeichert werden. Letzteres hat zwei Vorteile:
\begin{itemize}
\item Neu angelegte Daten gehen dem Nutzer nicht verloren, obwohl sie nicht zum Server geschickt werden.
\item Dem Nutzer bleiben Funktionalitäten und bereits vorhandene Daten erhalten, obwohl keine Verbindung zum Server besteht.
\end{itemize}
Fallen neue lokale Daten an, ergibt sich aus der Problembeschreibung, dass diese bei späterer Verbindung zum Server persistiert werden. Dies muss die Kommunikations-Schnittstelle durch einen geeigneten Synchronisationsmechanismus unterstützen. 

%Die vordergründige Herausforderung liegt darin, dass die Fitness-Anwendungen auch
%dann noch benutzbar sein sollen, wenn keine Verbindung zum Internet, speziell zum
%benötigten Server, besteht. Dafür müssen die Applikationen ausgelegt und vorbereitet werden. Sei es durch das Unterbinden von Funktionen oder das Speichern von bereits erhaltenen Daten, um diese dem Benutzer für die weitere Verwendung zur
%Verfügung stellen zu können. \\
%Weiterhin gibt es Unterschiede in der Auswahl der lokal zu speichernden Daten. Auf
%der einen Seite können alle Daten, die relevant sind, automatisch von der Anwendung für den Benutzer hinterlegt werden. Zum anderen kann es die Möglichkeit für den Benutzer geben, bestimmte Daten offline verfügbar zu machen.
%Zu beachten ist darüber hinaus, dass die Daten, die ohne Internetverbindung angelegt werden, wieder zum Server synchronisiert werden müssen, um Benutzereingaben zentral persistent speichern zu können. In diesem Anwendungsfall sollen Trainingsdaten erfasst und gespeichert werden.\\
%Die Daten sollen für verschiedene Benutzer, die sich an dem Gerät anmelden, gespeichert werden. Des Weiteren sollen Benutzer nur Funktionen ausführen können,
%zu denen sie auch autorisiert sind.\\
%Konkret kann daraus geschlossen werden, dass die Anwendungen mit einem Mechanismus
%ausgestattet sein müssen, der das lokale Zwischenspeichern von Informationen
%unterstützt. Damit soll das Abrufen von Daten im Offline-Modus ermöglicht
%werden. Des Weiteren soll es offline möglich sein, Daten anzulegen und diese sollen dann mit dem Server synchronisiert werden, wenn wieder eine Verbindung besteht.

\section{Soll-Konzept}
\label{sec:soll-konzept}
Aus der vorangehenden konkreten Problembeschreibung ergibt sich ein Soll-Konzept für die Umsetzung des Projekts, welches in den folgenden Abschnitten erläutert wird.
\subsection{Kommunikation zwischen Client und Server}
\label{ssec:kommunikation-client-server}
Ziel soll die Umsetzung zweier mobiler Applikationen sein, welche mit einem selbst entwickelten Server kommunizieren. Während der Kommunikation muss festgestellt werden, ob- bzw. wann diese abbricht. Abhängig davon müssen die Applikationen das Verhalten zwischen \textit{Online}- und \textit{Offline}-Modus umstellen. \\
Wenn der Server erreichbar ist, können die benötigten Daten dort direkt abgefragt und lokal angezeigt werden. Zum Entgegenwirken von Datenverlust für den Benutzer, werden die bei dieser Abfrage erhaltenen Informationen lokal gespeichert werden. Daten, die im \textit{Online}-Status angelegt werden, können direkt zum Server übertragen werden. Dort werden sie dann persistent gespeichert und sind für diesen Benutzer von überall erreichbar.\\
Wenn die Verbindung abgebrochen ist, können die Applikationen nur auf die abgespeicherten Daten zurückgreifen und diese anzeigen. Die Applikationen sollen die Möglichkeit bieten, auch im \textit{Offline}-Zustand Daten anzulegen. Geschieht dies, werden die Daten ebenfalls lokal gespeichert. Hierbei werden sie als \textit{Offline}-Daten mittels \textit{Flag} erkennbar in dem lokalen Speicher abgelegt.\\
Wenn die Verbindung zwischen Server und Client gerade wieder hergestellt werden kann, müssen lokal angelegte Daten zum Server übertragen werden. Zur Erkennung, welche Daten an den Server übertragen werden müssen, dient das \textit{Offline}-Flag aus den Daten des lokale Speichern. Bei dieser Übertragung muss eine Synchronisation der Daten erfolgen. 
\subsection{Evaluation zur Client-Entwicklung}
\label{ssec:evaluation-client-entwicklung}
Während der Umsetzung sollen zwei mobile Applikationen entwickelt werden.  Hierzu beschriebt die Aufgabenstellung die Implementierung in zwei unterschiedlichen Technologien. Dabei soll evaluiert werden, welche Technik für die Umsetzung der Anforderungen am Besten geeignet sind. Deshalb soll folgende Applikationen entwickelt werden: 
\begin{itemize}
\item eine mobil-optimierte Webseite (Web-App)
\item native App 
\end{itemize}
\subsubsection*{Single Page Application}
\label{ssec:aufgabenstellung:spa}
Die Web-App soll als \textit{Homepage} im Browser umgesetzt werden. Damit die clientseitige Logik einfacher umgesetzt werden kann, soll die Web-App als \textit{Single-Page-Application} (kurz \ac{SPA}) umgesetzt werden. Hierbei soll konsequent auf aktuelle Web-Techniken aus HTML5, CSS3 und Javascript gesetzt werden. \\
Da eine mobile Nutzung im Vordergrund steht, soll die SPA sich \gls{responsiv} verhalten. Dadurch wird eine Nutzung auf kleinen Displays unterstützt. Dies erhöht die Vergleichbarkeit der Applikationen, da beide Varianten problemlos auf dem gleichen Gerät getestet werden können.
\subsubsection*{Native App}
\label{ssec:aufgabenstellung:nat-app}
Die native App soll für Android entwickelt. Android wird als Plattform ausgewählt, um die Vorteile des offenen Systems nutzen zu können. So ist es beispielsweise möglich die entwickelten Apps ganz einfach auf einem Testsystem zu installieren, ohne - wie bei Apples iOS nötig - einen Entwickler-Account anlegen zu müssen.\\ Zudem ist es bei einer iOS-App notwendig, das Aufspielen einer Testapplikation über ein spezielles Entwickler-Tool in XCode durchzuführen. Diese Hürde fällt bei einer Android-App weg. Des Weiteren ist das Android-Betriebssystem weiter verbreitet (siehe \citep{Statista-SmartphoneVerteilung}). Dadurch kann bei einer möglichen späteren Weiterentwicklung eine größere Akzeptanz der App erzielt werden.
\subsection{Umsetzung der Client-Entwicklung}
\label{ssec:umsetzung-client-entwicklung}
Auf Grundlage der Evaluation soll eine der beiden Applikationen ausgewählt und anschließend zu einem rudimentären Messe-Prototypen weiterentwickelt werden. Diese soll eine komplette \gls{UserStory} implementieren. Als Beispiel soll eine Fitness-App dienen. Hierbei kann ein Nutzer Trainingsdaten verwalten. 
\section{Meilenstein-Plan}
\label{sec:meilenstein-plan}
Aus dem Soll-Konzept ergibt sich eine klare Aufteilung des Projekts in zwei Meilensteine:
\begin{table}[]
\centering
\caption{Meilensteinplan}
\label{tbl:meilensteinplan}
\begin{tabular}{|c|l|}
\hline
{\bf Meilenstein} & {\bf Titel}                                                      \\ \hline
1                 & Umsetzung der Clients als \textit\{Proof-of-Concept\}-Prototypen \\ \hline
2                 & Umsetzung einer mobilen Anwendung als Messeprototyp              \\ \hline
\end{tabular}
\end{table}

\subsection{Umsetzung der Clients als \textit{Proof-of-Concept}-Prototypen}
In diesem Meilenstein werden Erkenntnisse zur Implementierung einer verlässlichen mobilen Anwendung gesammelt. \\
Hierbei müssen folgende Teilschritte durchgeführt werden:
\begin{enumerate}
\item Erwerb grundsätzlicher Kenntnisse eines Caches und dessen Implementierung
\item Implementierung des Servers 
\item Erstellung einer Web-App
\item Erstellung einer nativen App
\item Evaluierung der Erkenntnisse
\end{enumerate}
Da beide Clients während der Implementierung den Server benötigen, müssen die Teilschritte bis einschließlich Schritt 2 synchron abgearbeitet werden. Die Umsetzung der beiden Clients kann anschließend parallel erfolgen. Dieser Meilenstein endet mit der Gegenüberstellung der gewonnen Erkenntnisse und der Auswahl einer Technik für Meilenstein 2.
\subsection{Umsetzung einer mobilen Anwendung als Messeprototyp}
In diesem Meilenstein wird die in Meilenstein 1 gewählte Technik benutzt, um einen Messeprototypen zu entwickeln. Hierbei sollen alle Funktionalitäten implementiert werden, um einen Anwendungsfall vollständig durchzuführen. Als Anwendungsfall soll ein Nutzer ein neues Trainingsdatum anlegen. Dabei soll es irrelevant sein, ob eine Verbindung zum Server besteht, oder nicht.
	\chapter{Grundlagen}
\label{cha:grundlagen}

\begin{itemize}
\item Wie funktioniert ein Cache?
\item Welche Arten gibt es ()?
\begin{itemize}
\item 80\% Store Forward 
\item 20\% Function Cache (klare Abgrenzung)
\end{itemize}
\item Sequendiagramme Caches
\end{itemize}

\begin{itemize}
\item 80\% zielführend
\item 20\% gefälliger Stil
\end{itemize}

	\chapter{Architektur}
\label{cha:architektur}
In diesem Kapitel werden die architektonischen Randbedingungen für die Entwicklung der Applikation beschrieben. Hierzu zählt, welche Anwendungsfälle in die späteren Prototypen und im schlussendlichen Messeprototyp gegeben sein muss. Daraus resultiert der Aufbau der Datenbank und die schlussendliche Systemarchitektur. Als Grundlage dient das Pflichtenheft. Dieses liegt dieser Arbeit gesondert im Anhang bei (siehe Anhang \ref{sec:Pflichtenheft}).

\section{Anwendungsfälle}
\label{sec:anwendungsfaelle}
Das Pflichtenheft sieht eine Unterteilung des Projekts in zwei aufeinanderfolgende Meilensteine vor (siehe Kapitel \ref{sec:meilenstein-plan}). Hierbei werden erst Proof-of-Concept-Prototyp entwickelt. Anschließend wird ein Prototyp zum Messeprototyp weiterentwickelt. \\
Für diese beiden Prototypen müssen andere bzw. erweiterte Anwendungsfälle implementiert werden. Darum werden nachfolgend für die beiden Implementierungsschritte die Anwendungsfälle einzeln aufgeschlüsselt. 
\subsection{Anwendungsfälle für Meilenstein 1 (Proof-of-Concept-Phase)}
\label{ssec:anwendungsfaelle-poc}
Aus dem Pflichtenheft ergeben sich folgende Anwendungsfälle für die erste Phase des Projekts:
\begin{itemize}
\item Es soll möglich sein, sich an der Anwendung anzumelden
\item Es soll möglich sein, eine Entität mit Daten (Trainingsplan, Training oder Übung), unabhängig von der Verbindung zum Web Service, persistent anzulegen, zu ändern und zu speichern
\item Optional soll sich ein Nutzer an der Anwendung registrieren können
\end{itemize}

\begin{figure}[h]
\centering
\includegraphics[width=0.8\linewidth]{content/images/UseCase-Proof-of-Concept.png}
\caption{Use-Cases Proof-of-Concept}
\label{pic:usecase-poc}
\end{figure}

\subsection{Anwendungsfälle für Meilenstein 2 (Messeprototyp-Phase)}
\label{ssec:anwendungsfaelle-messe}
Für den Meilenstein 2 werden die bereits vorgestellten Anwendungsfälle weiter verfeinert. Daraus ergeben sich folgende Anwendungsfälle:
\begin{itemize}
\item Ein Nutzer soll sich an der Anwendung anmelden können.
\item Ein Nutzer soll seine eigenen Trainingsplan-Daten abrufen können
\item Ein Nutzer soll zu einem seiner Trainingspläne alle zugehörigen Übungen abrufen können
\item Ein Nutzer soll zu einer dieser Übungen seine bisherigen Trainingsdaten abrufen können
\item Ein Nutzer soll zu einer Übung ein neues Training anlegen können
\item Alle nicht-optionalen Anwendungsfälle müssen unabhängig von einer Serververbindung funktionieren und eventuell anfallende Daten dauerhaft speichern.
\item Optional: Ein Nutzer soll eine Statistik der letzten Trainings zu einer Übung abrufen können. Neu erstellte Trainingsdaten aktualisieren diese Statistik
\item Optional: Ein Nutzer soll sich an der Applikation registrieren können
\item Optional: Ein Nutzer mit der Rolle \textit{Administrator} soll neue Übungen anlegen können
\end{itemize}

\begin{figure}[h]
\centering
\includegraphics[width=0.8\linewidth]{content/images/UseCase-Messeprototyp.png}
\caption{Use-Cases Messeprototyp}
\label{pic:usecase-messe}
\end{figure}
\newpage
\section{Datenbank-Entwurf}
\label{sec:Datenbank-Entwurf}
Aus den definierten Anwendungsfälle ergibt sich die Struktur für die Datenbank. Als Grundlage werden die Anwendungsfälle des zweiten Meilensteins genutzt, um spätere Anpassungen nach Beendigung des ersten Meilensteins zu vermeiden.

\begin{figure}[h]
\centering
\includegraphics[width=0.8\linewidth]{content/images/DB-Entwurf.png}
\caption{Datenbank-Entwurf}
\label{pic:db-entwurf}
\end{figure}

\section{Programmarchitektur}
\label{sec:programmarchitektur}
Da verschiedene Clients implementiert werden sollen, ist es sinnvoll das Projekt als Mehrschichtenarchitektur für verteilte Anwendung zu implementieren. \\
Der Server hält dabei die Funktionen zur Nutzung durch die Clients vor. Konkret greift der Server per \gls{OR-Mapper} auf die Datenbank zu, bereitet die Daten in der Applikationsschicht auf und reicht sie über eine \ac{REST}-Schnittstelle an die anfragenden Client weiter. Eine detaillierte Beschreibung zu \ac{REST} wird in Kapitel \ref{sec:definition-rest} gegeben. Bei dieser Kommunikation muss gewährleistet sein, dass ein Nutzer nur die Daten abrufen darf, für die er autorisiert wurde. \\
Clientseitig werden Daten in einer Caching-Schicht zum Schutz vor eventuellen Verbindungsabbrüchen zwischengespeichert. Anschließend werden die erhaltenen Daten auf dem Endgerät für die Anzeige aufbereitet und angezeigt. \\
Abbildung \ref{pic:architecture} bildet diesen Aufbau grafisch ab:
\begin{figure}[h]
\centering
\includegraphics[width=0.7\linewidth]{content/images/Aufbau-Architektur.png}
\caption{Aufbau der Anwendung}
\label{pic:architecture}
\end{figure}

\newpage
\section{Rollen-Konzept}
\label{sec:rollen-konzept}
Im Pflichtenheft wird für den zweiten Meilenstein eine Unterscheidung in den Berechtigungen gemacht. So dürfen beispielsweise nur Administrationen neue Übungen anlegen. Darum ist es nötig, ein Rollenkonzept zu entwickeln, der den Zugriff auf bestimmte Ressourcen reguliert. \\
Aus den Aussagen, die im Pflichtenheft getroffen wurden, geht hervor, das sich der Nutzer in einer der nachfolgenden Status befindet, wenn er auf den Web Service zugreifen will: 
\begin{itemize}
\item unautorisiert \\
Der Nutzer hat sich noch nicht gegenüber des Web Server authentifiziert. In diesem Status kann der Nutzer sich mit seinen Anmeldedaten einloggen oder als neuer Nutzer an der Web Applikation registrieren.
\item Rolle \textbf{Nutzer}\\
Jeder angemeldete Nutzer besitzt die Rolle \textit{Nutzer}. Ein normaler Nutzer kann seine Daten einsehen. Dies beinhaltet seine Trainingspläne, deren Übungen und die dazu angelegten Trainingseinheiten.
\item Rolle \textbf{Administrator}\\
Ein \textit{Administrator} ist ebenfalls ein Nutzer. Er kann zusätzlich neue Übungen anlegen. 
\end{itemize}
	\chapter{Aspekte der Realisierung}
\label{cha:realisierung}

\section{Entwicklungsumgebung}
\label{sec:entwicklungsumgebung}

\section{DB-System}
\label{sec:DB-System}

\section{Hosting-Plattform}
\label{sec:Hosting-Plattform}

\section{Testing (evtl)}
\label{sec:Testing}	
	\chapter{Realisierung der serverseitigen Implementierung}
\label{cha:server-impl}
In diesem Kapitel wird näher auf die Implementierung des in Kapitel \ref{sec:programmarchitektur} besprochenen Webservices eingegangen. Es enthält eine Übersicht über die genutzten Komponenten und die konkreten Techniken, welche für die Implementierung genutzt wurden. Anschließend wird gesondert auf Sicherheitsaspekte in Verbindung mit \ac{REST}ful-Architekturen eingegangen. Der hier beschriebene Web Service kann über die \ac{URL} \href{http://fit-bachelor.azurewebsites.net/}{http://fit-bachelor.azurewebsites.net/} erreicht werden. 
\section{Was ist ein Webservice?}
\label{sec:definition-webservice}
Um verteilte Systeme aufzubauen ist es nötig, eine Struktur zu implementieren, mit der Maschinen untereinander kommunizieren können. Diese Aufgabe übernehmen Webservices. Sie stellen innerhalb eines Netzwerkes Schnittstellen bereit, damit Maschinen plattformübergreifend Daten austauschen können. Hierbei wird meistens \ac{HTTP} als Träger-Protokoll genutzt, um eine einfache Interoperabilität zu gewährleisten.\footcite{Definition-Webservice} Die dabei angeforderten Daten werden in der Regel im \textit{\ac{XML}}- oder \textit{\ac{JSON}}-Format übermittelt. 
\subsection{Besonderheiten eines RESTful Webservices}
\label{sec:definition-rest}
Da Webservices in der Regel \ac{HTTP} als Protokoll verwenden, wurde die Idee zur Implementierung eines Webservices erweitert, um die Möglichkeiten des Protokolls besser zu benutzen. Daraus entstand das Programmierparadigma \ac{REST}. Mit einem \ac{REST}-Server bzw. einem \ac{REST}ful Webservice bezeichnet man einen Webservices, welcher die strikte Nutzung von \ac{HTTP} als Programmierparadigma umsetzt.  Dies meint, dass sich, wie im Internet üblich, \ac{URI} zur eindeutigen Identifikation von Ressource genutzt werden. Nachfolgend werden einige Prinzipien von \ac{REST} näher beleuchtet.

\subsubsection*{Adressierbarkeit}
Im Gegensatz zu anderen Webservice-Implementierungen stellen \ac{REST}ful Webservices keine Methoden oder aufrufbare Funktionalitäten zu Verfügung, sondern ausschließlich Ressourcen. Dies hat den Vorteil, dass die Schnittstelle leicht und eindeutig beschrieben werden kann, da ein Aufruf einer \ac{URL} an den \ac{REST}-Service immer eindeutig auf eine Ressource zeigt, ohne das Abhängigkeiten oder ein Kontext berücksichtigt werden müssen. \\
In den meisten Fällen, wie auch in den Anwendungsfällen dieser Arbeit, soll der Webservice \textit{\ac{CRUD}}-Funktionalitäten bereitstellen. Damit die Schnittstelle nicht durch unnötig viele unterschiedliche \ac{URL}s aufgebläht wird, sieht der \ac{REST}ful-Ansatz die Verwendung der verschiedenen \ac{HTTP}-Verben vor, um mit den Ressourcen zu Interagieren. Auf die Nutzung von Http-Verben wird im nächsten Abschnitt näher eingegangen. Zur Interaktion mit Ressourcen werden zwei Arten von \ac{URL}s unterschieden, um in Kombination mit \ac{HTTP}-Verben verschiedene Aufgaben zu erfüllen. Zur Veranschaulichung sollen uns folgende zwei \ac{URL}s dienen:
\begin{itemize}
\item http://myRestService.de/Schedule
\item http://myRestService.de/Schedule/123
\end{itemize}
Es fällt auf, dass die beiden \ac{URL}s sich bis auf das letzte Segment gleichen. Im ersten Fall wird die \ac{URI} als \textit{Collection \ac{URI}} bezeichnet, da hiermit die Gesamtheit aller Trainingspläne angesprochen wird. Im zweiten Fall wird die ID einer Trainingsplans benutzt und mit einer konkreten Trainingsplan-Ressource zu interagieren. Man spricht hier von einer \textit{Element \ac{URI}}.\footcite[S. 12ff.]{Building-a-REST-Service} Diese können mit verschiedenen \ac{HTTP}-Verben kombiniert werden.\footcite[S. 26ff.]{REST-und-HTTP}
\subsubsection*{Nutzung von HTTP-Verben}
Um den Rahmen der Arbeit nicht zu überspannen, wird sich hier nur auf die Vorstellung der vier meist verwendeten \ac{HTTP}-Verben beschränkt:\\
Das Verb GET ruft eine Ressource vom Server ab, wobei diese nicht verändert wird. Bei Nutzung einer \textit{Collection \ac{URI}}, werden alle Einträge dieser Entität als Verbundstruktur abgerufen. Jedes Element der Struktur beinhaltet die \textit{Element URI} auf das konkrete Element. Wird GET auf eine \textit{Element URI} aufgerufen, wird das konkrete Objekt aufgerufen. Hierbei antwortet der Server, dem \ac{HTTP}-Standard folgend, mit dem Status-Code \textit{200 (OK)} bei erfolgreicher Suche oder \textit{404 (Not Found)}, wenn keine Ressource gefunden wurde.\\
Das POST-Verb wird zur Erstellung neuer Inhalte verwendet. Bei Nutzung von \textit{Element \ac{URI}s} wird versucht, die ID für das neue Element zu benutzten. In der Regel wird das ID-Management aber auf dem Server implementiert, sodass eine Collection \ac{URI} zur Erstellung von Elementen zum Einsatz kommt.\\
Mit dem HTTP-Verb PUT wird eine vorhandene Ressource geändert oder hinzugefügt. Obwohl es \ac{REST}-conform wäre, eine Collection \ac{URI} per PUT aufzurufen, wird dies selten implementiert, da der normale Anwendungsfall ist, dass ein einzelnes Objekt geändert werden soll. Stattdessen wird sich auf \textit{Element \ac{URI}s} beschränkt. Ist eine Ressource mit der übergebenen ID nicht vorhanden, wird je nach Implementierung entweder ein neues Objekt mit der ID erstellt (\textit{Statuscode 201 (Created)}) oder die Verarbeitung verweigert. Der Server gibt dann den Statuscode \textit{400 (Bad Request)} oder \textit{404 (Not Found)}  zurück. \\
Das letzte \ac{HTTP}-Verb, welches an dieser Stelle vorgestellt werden soll, ist DELETE. Wie der Name vermuten lässt, wird damit eine Ressource vom Server entfernt. Wie auch bei PUT wird in der Regel auf eine Implementierung von DELETE als \textit{Collection \ac{URI}} verzichtet, da sonst alle Einträge einer Entität gelöscht werden können. Im Erfolgsfall wird mit dem Statuscode \textit{200 (Ok)} geantwortet und bei Fehlern mit \textit{400 (Bad Request)} oder \textit{404 (Not Found)}.\footcite[S. 26ff.]{REST-und-HTTP}
\subsubsection*{Zustandslosigkeit}
Das zum Datenaustausch genutzte Protokoll \ac{HTTP} ist statuslos. Das bedeutet, dass kein Kontext bei der Kommunikation besteht bzw. jede Kommunikation unabhängig von vor- oder nachherigen Verbindungen ist. Deshalb muss ein \ac{REST}ful Webservice so implementieren werden, dass alle Informationen, welche für die Kommunikation notwendig sind, bei jeder Anfrage mitgesendet werden. Was vordergründig als Nachteil erscheint, ist ein wesentlicher Vorteil. Dadurch, dass jeder Request alle nötigen Informationen mitliefert, ist es nicht nötig, Kontext der Kommunikation über mehrere Requests auf dem Server zu verwalten. Dadurch kann ein \ac{REST}ful Webservice sehr leicht skaliert werden.\footcite[S. 26ff.]{REST-und-HTTP}
\subsubsection*{Daten sind unabhängig von der Präsentation}
Das \ac{REST}ful-Paradigma besagt, dass Daten losgelöst von einer Repräsentation bereit gestellt werden. Darum ist ein \ac{REST}ful Webservice so zu implementieren, dass der Client das gewünschte Datenformat anfragen kann. Bei Nutzung des Protokolls HTTP wird dies in der Regel über die Header-Eigenschaft \textit{accept} realisiert, welche gewünschten Datenformate angibt. Wird dieses nicht vom Server unterstützt, werden die angeforderten Daten in einem Standard-Format zurückgegeben. \footcite[S. 26ff.]{REST-und-HTTP}
\section{Aufbau der Komponenten}
\label{sec:aufbau-Komponenten}
In diesem Abschnitt wird beschrieben, wie der zuvor theoretisch beschriebene \ac{REST}-Ansatz für das Projekt umgesetzt wurde. Der aus der Umsetzung entstandene Server besteht aus zwei Teilen: Der Datenbank und der Web \ac{API}, welche jeweils gesondert vorgestellt werden. \\
Die Web \ac{API} bietet eine Schnittstelle zum Abrufen der Daten mittels \ac{HTTP}. Diese wurde nach dem Design-Pattern \textit{\ac{MVVM}} aufgebaut. Die erzeugten Objekte, welche Tupel einer Datenbank-Relation darstellen, werden aus speziell dafür präparierten Model-Klassen erzeugt. Bevor diese Daten dann über die Web \ac{API} bereitgestellt werden, werden sie vom Model in ein ViewModel übertragen. Hierbei wird, nach dem Grundgedanken des \gls{Seperation-of-Concerns}, klar zwischen den Models für die Datenbank und den ViewModels, welche die Web \ac{API} benutzt, unterschieden.
\subsection{Datenbank}
\label{ssec:aufbau-server-db}
Wie bereits in Kapitel \ref{sec:DB-System} vorgestellt, wurde die Datenbank-Lösung \textit{\ac{MSSQL}} von \textit{Microsoft} zur Datenhaltung gewählt. Dies hat den Vorteil, dass das \textit{\gls{MSEF}}, welches sehr gut für die Nutzung mit einer Web \ac{API} optimiert ist, als \gls{OR-Mapper} genutzt werden kann. Dieser bietet das Design-Pattern \textit{Code First}. Das bedeutet, dass anhand präparierter Model-Klassen die benötigten Relationen in der Datenbank automatisch erzeugt wird. \footcite{entity-framework-code-first}\\
An den folgenden Beispielen wird exemplarisch beschrieben, wie die Model-Klassen aufgebaut wurden und wie sich daraus die Struktur der Datenbank ergibt. Grundlage für Model-Klassen ist das Interface \textit{IEntity}(Quellcode \ref{lst:IEntity}):
\lstinputlisting[caption=Basisinterface für DB-Repräsentationen, label=lst:IEntity, style=sharpc]{content/listings/IEntity.cs}
Das Interface gewährleistet, dass jede Datenbank-Entität einen eindeutigen Schlüssel besitzt.
Eine konkrete Implementierung für eine Model-Klasse sieht man im \linebreak Quellcode-Beispiel \ref{lst:Schedule}, in der die Trainingspläne implementiert sind:
\lstinputlisting[caption=Modelklasse für Trainingspläne, label=lst:Schedule, style=sharpc]{content/listings/Schedule.cs}
Hierbei zeigt sich gut, was mit einer präparierten Klasse gemeint ist. Über die Annotation \textit{Required} wird definiert, dass die Eigenschaft \textit{Name} zwingend bei Insert- und Update-Operationen gesetzt werden muss. \\
Gleichzeitig sieht man an diesem Beispiel, wie das Entity Framework über Namenskonventionen Verbindungen zwischen Entitäten auflöst. Auf Grund des Aufbaus der Klasse \textit{Schedule} wird eine einwertige Fremdschüssel-Beziehung zu der Model-Klasse \textit{User} erzeugt, da folgende Bedingungen erfüllt sind:
\begin{itemize}
\item Die Klasse \textit{User} besitzt eine Eigenschaft \textit{ID} vom Datentyp \textit{string}
\item Die Klasse \textit{Schedule} besitzt eine Eigenschaft \textit{UserID} vom Datentyp \textit{string}
\end{itemize}
Auch die Erstellung einer mehrwertigen Beziehung lässt sich aus dem Code-Beispiel \ref{lst:Schedule} ablesen: \\
Da es eine Entität gibt, welche \textit{Exercise} heißt und die Model-Klasse \textit{Schedule} eine Verbundstruktur besitzt, welche \textit{Exercises} heißt, wird implizit eine Verbindung zwischen den Relationen in der Datenbank angelegt. \footcite{entity-framework-code-first}
\subsection{Web API}
\label{ssec:aufbau-webapi}
Die Umsetzung der \ac{REST}-Schnittstelle wurde mit Hilfe des \textit{Microsoft}-Frameworks \textit{ASP.NET Web API 2} (kurz \textit{Web \ac{API}-Framework}) realisiert. Dieses ermöglicht es, Controller-Methoden zu erstellen, welche über definierte Routen per \ac{HTTP} aufgerufen werden können. Hierbei wird die Umsetzung im Sinne des \ac{REST}-Paradigmas durch vorhandene Funktionen unterstützt.\footcite[S. 2ff.]{Building-a-REST-Service}\\
Dies wird beispielhaft an der Methode aus  Quellcode-Abbildung \ref{lst:ScheduleController} gezeigt:
\lstinputlisting[caption=POST-Methode zur Erstellung eines Trainingsplans, label=lst:ScheduleController, style=sharpc]{content/listings/SchedulesController.cs}
Im Beispiel fällt auf, dass das \textit{Web \ac{API}-Framework} die Nutzung von Annotationen fördert: Das Routing kann durch die Annotationen \textit{Route} (Zeile \ref{line:SchedulesController_Route}) an der Methode und \textit{RoutePrefix} (Zeile \ref{line:SchedulesController_RoutePrefix}) am gesamten Controller konfiguriert werden. Neben der Konfiguration der Route muss dem Framework noch mitgeteilt werden, welche \ac{HTTP}-Verben in dieser Methode zulässig sind. Das \textit{Web \ac{API}-Framwork} bietet hierfür pro Verb eine eigene Annotation. Im Codebeispiel \ref{lst:ScheduleController} wird über die Annotation \textit{HttpPost} (Zeile \ref{line:SchedulesController_HTTPVerb}) ausgesagt, dass nur POST-Request durch diese Methode verarbeitet werden.\footcite{webApi-AttributeRouting} \\
Das Framework versucht die empfangenen Daten in einem ViewModel-Objekt zu kapseln und anschließend zu validieren. Die dafür genutzten Validatoren werden direkt im View-Model als Annotationen angegeben.\footcite{webApi-Validation} Die Klasse \textit{EntryModel} (Beispiel \ref{lst:EntryModel}) zeigt die Möglichkeit in Zeile \ref{line:EntryModel_Annotation_1} und \ref{line:EntryModel_Annotation_2}. \\ 
\lstinputlisting[caption=Basis-Model-Klasse, label=lst:EntryModel, style=sharpc]{content/listings/EntryModel.cs}
Schlägt die Validierung fehl, werden die Fehler mit dem passenden Statuscode zurückgegeben. Andernfalls werden die Daten per \gls{Factory}-Klasse in ein Model konvertiert und per \gls{Repository}-Klasse in der Datenbank persistiert. Anschließend wird dem ViewModel, im Sinne des \ac{REST}-Gedankens, ein URL zur GET-Methode mit der ID des neu erstellten Objekts übergeben. 
\subsubsection*{Swagger}
\label{sssec:Swagger}
Da die Web \ac{API} zur Entwicklung der Clients benötigt wurde, wurde schnell die Notwendigkeit einer Dokumentation des aktuellen Stands klar. \\
Aus diesem Grund wurde \textit{Swagger} in die Web \ac{API} integriert. \textit{Swagger} ist ein quelloffenes Framework zur Dokumentation von \ac{REST}ful Web \ac{API}s, welche von vielen großen Konzernen genutzt wird\footcite{swagger}. Durch Nutzung des \gls{NuGet}-Packets \textit{Swashbuckle} konnte durch hinzufügen von Kommentaren und Annotationen eine vollständige und übersichtliche Dokumentation erstellt werden\footcite{implementing-Swagger}. Die Abbildung \ref{pic:swagger-UI} zeigt diese. \\
Da das Authorisierungsprotokoll OAuth in Version 2 (kurz: \textit{OAuth2})(näheres in Kapitel \ref{ssec:oauth2}) zum Durchführungszeitpunkt des Projekts noch nicht von \textit{Swagger} unterstützt wird, kann das Ausführen von API-Request aus \textit{Swagger} heraus nur für Methoden durchgeführt werden, für die keine Autorisierung des Nutzers benötigt wird. \\
Die Dokumentation ist unter \href{http://fit-bachelor.azurewebsites.net/swagger}{http://fit-bachelor.azurewebsites.net/swagger} aufrufbar. 
\begin{figure}[h]
\centering
\includegraphics[width=0.8\linewidth]{content/images/Swagger-UI-fIT}
\caption{Swagger UI der Web Api}
\label{pic:swagger-UI}
\end{figure}


\section{Authentifizierung \& Autorisierung}
\label{sec:server-authorisierung}
Wie bereits in Kapitel \ref{sec:rollen-konzept} beschrieben, darf nicht jeder Nutzer auf alle Daten zugreifen. Um dies zu bewerkstelligen, wurde ein Login-Mechanismus implementiert, welcher bekannte Nutzer authentifiziert. Da jedoch nicht alle authentifizierten Nutzer alle bereitgestellten Web \ac{API}-Methoden benutzen dürfen wurden auf Basis des \textit{\ac{RBAC}} Rollen implementiert, welche den Nutzer zur Nutzung verschiedener Aufrufe autorisieren. Zur Umsetzung dieser Anforderungen wurde das Protokoll \textit{OAuth2} implementiert.\footcite{online:WebApi_Authorize}
\subsection{OAuth2}
\label{ssec:oauth2}
OAuth2 ist ein Protokoll zur Authentifikation und zur Delegation von Zugriffsrollen. Die Struktur von OAuth2 kennt vier Instanzen, welche diesem Vorgang miteinander kommunizieren, nämlich \textit{Client}, \textit{Resource Owner}, \textit{Authorization Owner} und \textit{Resource Server}.\footcite[S. 286]{book:AngularJs:Steyer2015} 
\begin{figure}[h]
\centering
\includegraphics[width=1\linewidth]{content/images/OAuth2}
\caption{Resourcenzugriff durch OAuth2}
\label{pic:OAuth2}
\end{figure}
\subsubsection*{Client}
Der \textit{Client} ist ein Endpunkt, welcher eine Ressource (beispielsweise Trainingspläne) abrufen möchte. In unseren Fall ist das die Web- oder die native App. Diese kommunizieren jeweils mit den anderen Instanzen.
\subsubsection*{Resource Owner}
Der \textit{Resource Owner} ist, wie der Name schon sagt, der Besitzer der geforderten Ressource. Der \textit{Client} erfragt im ersten Schritt beim \textit{Resource Owner} den Zugriff zu einer Ressource.\\
Im diesem Projekt registriert sich der Nutzer an der Web \ac{API}. Anschließend kann er unter seinem Account Daten (Trainingspläne und Trainings) anlegen. Diese angelegten Daten sind die geforderten Ressourcen. Da diese vom Nutzer selbst angelegt wurden, erhält er automatisch die Erlaubnis (\textit{Grant}) zur Anfrage am \textit{Authorization Server}\footcite{online:Implemented_OAuth_WebToken}.
\subsubsection*{Authorization Server}
\label{sssec:authorization-server}
Der Nutzer meldet sich nun mit der erhaltenen Erlaubnis am \textit{Authorization Server} an. Dieser hat Kenntnis über alle vorhandenen Nutzer und deren Rollen\footcite{online:Implemented_OAuth_Roles}. Bei erfolgreicher Anmeldung erhält der Nutzer ein kurzlebiges \textit{Access-Token}, dem Typen des Access-Tokens, dessen Ablaufdatum und ein langlebiges \textit{Refresh-Token}. Das Access-Token wird im nächsten Schritt benutzt, um die gewünschte Ressource anzufordern. Das \textit{Refreh-Token} wird benutzt, um ein neues \textit{Access-Token} anzufordern. Die beiden Token-Arten werden nochmal genauer in Abschnitt \ref{ssec:jwt-bearer} besprochen.\footcite[S. 287]{book:AngularJs:Steyer2015} 
\subsubsection*{Resource Server}
Der \textit{Resource Server} enthält die geforderten Ressourcen. Ab dieser Anfrage muss das Access-Token bei jeder Anfrage mitgesendet werden. Konkret passiert dies, indem im Header der Anfrage um den Schlüssel \textit{authorization} erweitert wird.

Durch diese strikte Trennung dieser Instanzen ist es ohne weiteres möglich, dass unterschiedliche Systeme die jeweiligen Aufgaben übernehmen. Daraus hat sich in letzter Zeit etabliert, dass es immer häufiger \textit{Single-Sign-On}-Szenarien implementiert werden. Dabei muss ich der Nutzer nur an einer Stelle registrieren (z.B. Bei Facebook oder Twitter). Will der Nutzer nun auf eine andere Ressource zugreifen, kann der Ressource-Server ein Access-Token vom Facebook-Authorisierungsserver akzeptieren. Dies hat für den Nutzer den Vorteil, dass er sich nicht bei mehreren Seiten registrieren muss, sondern jedes mal Zugriff über den Authorisierungsserver mithilfe seiner Credentials erhält. \footcite[S. 294]{book:AngularJs:Steyer2015} 
\subsection{JWT and Bearer Token}
\label{ssec:jwt-bearer}
Sowohl das Access-Token als auch das Refresh-Token sind \ac{JWT}. Das sind codierte und meistens auch signierte Repräsentationen von Daten. Zur genaueren Betrachtung des Aufbaus, wird folgend ein Access-Token näher beschrieben. Es besteht aus 3 Teilen, welche mit einem Punkt voneinander getrennt sind. Die Daten selber sind mithilfe des Kodierungsverfahren \textit{base64} verschlüsselt.\footcite[S. 289]{book:AngularJs:Steyer2015} Die Bestandteile sind:
\begin{itemize}
\item \textbf{Header}\\Hier wird der Typ des Tokens und der Algorithmus, welcher für die Verschlüsselung benutzt wurde, angegeben. 
\item \textbf{Payload} \\Die zu übermittelnden Daten werden als \ac{JSON}-Objekt bereitgestellt. Das Objekt enthält sowohl die Informationen für die Kommunikation, wie beispielsweise den Nutzername und Rollen, als auch Meta-Daten über das Token (z.B. das Ablauf-Datum).
\item \textbf{Signatur}\\Damit gewährleistet ist, dass die Daten unverändert wurden, werden Sie mit einem Client-Secret verschlüsselt. Dies bedeutet aber auch, dass der Server jeden Client kennen muss, welcher sich beim \textit{Authorization Server} anmelden will. \\Da es sich bei diesem Projekt um einen Prototypen handelt, wurde die Implementierung der Client-Verwaltung nicht durchgeführt, da es für den Ablauf nicht zwingend benötigt wird. Der Server lässt alle gültigen Access-Token und alle bekannten Refresh-Tokens zu. Im produktiven Einsatz müsste diese Komponente dringend nachträglich implementiert werden, da sonst eine Sicherheitslücke entsteht.\footcite{online:understanding-jwt}
\end{itemize}
Wie bereits im Abschnitt zum \textit{Authorization Server} (siehe \ref{sssec:authorization-server}) beschrieben, wird für das \textit{Access-Token} eine recht kurze- und für das \textit{Refresh-Token} eine sehr lange Lebenszeit gewählt. Dies hat zwei Vorteile:\\
Das \textit{Access-Token} wird bei Request an den Server mitgesendet. Sollte das Token von Dritten abgefangen werden, können diese nur für kurze Zeit im Namen des Nutzers Aktionen durchführen. Das Abgreifen eines solchen Tokens wird im produktiven Gebrauch durch zusätzliche Sicherheitsmaßnahmen, wie die Nutzung des Protokolls \textit{\ac{HTTPS}} erschwert. \\
Da das \textit{Refresh-Token} ausschließlich zum Erneuern des Access-Tokens benutzt wird, ist die Gefahr, dass es abgefangen wird wesentlich geringer, wodurch die lange Lebensdauer vertretbar ist. \\
Außerdem bleiben durch die kurze Lebensdauer des \textit{Access-Tokens} die Daten immer aktuell. Sollte sich an den Daten des Nutzers etwas ändern (z.B. wird eine Rolle hinzugefügt oder entzogen) wird diese Änderungen beim nächsten Abrufen eines \textit{Access-Tokens} in die Payload codiert\footcite{online:Implemented_OAuth_RefreshToken}. Somit ist immer gewährleistet, das der Nutzer nur die Funktionalität nutzt, für die er auch autorisiert ist. 
\subsection{Zugriff per CORS}
\label{ssec:cors}
Im vorherigen Abschnitt wurden Maßnahmen beschrieben, damit nur autorisierte Nutzer an geschützte Daten herankommen. Mit \textit{\ac{CORS}} wird ein weiterer Mechanismus vorgestellt, welcher den Zugriff auf die Daten per \textit{\ac{AJAX}} beschränkt. \\
Um den Nutzer davor zu schützen, dass eine Webseite im Hintergrund Daten von anderen Quellen nachlädt, ist in jedem Browser eine \textit{Same-Origin-Policy} implementiert. Diese besagt, dass nur Daten aus der gleichen Domäne, aus der der \textit{\ac{AJAX}}-Aufruf abgesetzt wurde, abgerufen werden dürfen. \\
Da es trotzdem häufig nötig ist, auf fremden Domains zuzugreifen, wurden schnell Hilfskonstrukte wie das Vorgehensmodell \textit{\gls{JSONP}} eingeführt. Da diese jedoch von vielen Entwicklern als nicht elegant empfunden wurden\footcite[S. 102]{book:AngularJs:Steyer2015}, wurde mit \ac{CORS} ein standardisierter Weg entwickelt, um Daten von fremden Domains abzurufen. \\
Hierbei wird beim Server eine Liste an gültigen Domains für eine domainübergreifende Anfrage hinterlegt. Soll nun vom Browser eine Anfrage an den Server gesendet werden, wird über das \ac{HTTP} Verb unterschieden, ob durch diese Anfrage eine Server-Datum verändert wird. Dies geschieht bei PUT, DELETE und POST, wobei letzteres eine Ausnahme bildet. Werden per POST Daten in einem Format übermittelt, welches beim Absenden eines Formulars genutzt wird (z.B. \textit{application/x-www-form-urlencoded}), wird die Anfrage wie ein nicht-ändernder Aufruf behandelt. \\
Wenn nun eine Daten-Änderung im Sinne von \ac{CORS} durch den Aufruf angestoßen wurde oder wenn der Aufruf zusätzliche Schlüssel im Header enthält, wird vor der Ausführung ein \textit{Preflight} gesendet. Dies ist eine OPTIONS-Anfrage, welche genutzt wird, um die Durchführung der bevorstehenden Anfrage zu validieren.
Enthält die Antwort im Header nicht den Schlüssel \textit{Access-Control-Allow-Origin} mit der aufrufenden Domäne, wird vom Browser ein Fehler erzeugt. Andernfalls wird die Abfrage an den Server gesendet\footcite[S. 102]{book:AngularJs:Steyer2015}. Dadurch ist gewährleistet, dass nur berechtigte Clients anfragen an den Server senden. Es wurde auf weitere Implementierung von \gls{Polyfills} verzichtet, da \ac{CORS} bereits in allen modernen Browsern genutzt werden kann\footcite{online:can-i-use:cors}.
\section{Testen der Funktionalität}
\label{sec:server-tests}
Die erwartete Funktionsweise des Servers ist Grundvoraussetzung für die Umsetzung der Clients. Um diese zu bewerkstelligen wurde die \textit{ManagementApi} entwickelt. Dies ist ein portable \ac{DLL}, welche alle Anfragen an den Server in Methoden kapselt. Hierbei wurde bei der Erstellung der \ac{DLL} darauf geachtet, dass sie sowohl in klassischen Testprojekten als auch zur Umsetzung der nativen App genutzt werden kann. Darüber hinaus war ein Design-Ziel, dass alle Methode auch asynchron aufrufbar sind, damit Nutzer der \ac{DLL} in der Erstellung Ihres Programmablaufes größere Flexibilität besitzen. \\
Die erzeugten Methoden wurden iterativ entwickelt und direkt getestet. Hierbei wurden zur Verifikation der Funktionalität ausschließlich Positivtests erstellt. So war sichergestellt, dass jede Methode der \textit{ManagementApi} die gewünschten Aktionen auf dem Webservice durchführt. \\
Der nachfolgende Codeausschnitt zeigt beispielhaft die Entwicklung eines Testfalls:
\lstinputlisting[caption=Implementierung des Tests 'Nutzer kann eigene Daten anpassen', label=lst:ClientTests.Users, style=sharpc]{content/listings/ClientTests.Users.cs}
Dadurch, dass die Klassen \textit{ManagementService} und \textit{ManagementSession} das Interface \textit{IDisposable} implementieren, kann für jeden Test unabhängig eine neue Session erstellt werden, in der der Test läuft. Ist der \textit{using}-Block vollständig durchlaufen, wird die Dispose-Methode aufgerufen, welche die verwendeten Ressourcen wieder freigibt (siehe Zeile \ref{line:ClientTests:Disposable}f.). 
	\chapter{Realisierung der clientseitigen Implementierung als native App}
\label{cha:native-app}
Dieses Kapitel widmet sich der Implementierung der nativen Applikation. Im Kapitel \ref{cha:architektur} wurde eine grobe Übersicht zu der Umsetzung und der Funktionsweise dieser \ac{App} gegeben, die nun verfeinert wird.
Dabei werden folgend die verwendeten Komponenten und Techniken erläutert und die Zusammenhänge zwischen den Techniken dargestellt.
\section{Allgemeine Funktionsweise einer Android-App}
\label{sec:definition-android}
Grundlegend für die Entwicklung einer Android-App ist das Wissen über die Basis des Systems, auf dem entwickelt wird. 
Bei dem Betriebssystem \ac{Android} handelt es sich um eine Art eines \ac{monolithisch}en Multiuser-\ac{Linux}-Systems. \footcite{Android-Fundamentals}
Dieses Betriebssystem stellt die Hardwaretreiber zur Verfügung und führt die Prozessorganisation, sowie die Benutzer- und Speicherverwaltung durch.
Jede Applikation wird in einem eigenen Prozess gestartet. In diesem Prozess befindet sich eine \ac{Sandbox}, die eine virtuelle Maschine mit der Applikation ausführt. Die Kommunikation aus der Sandbox heraus kann nur über Schnittstellen des Betriebssystems geschehen. Diese Einschränkung sorgt für Sicherheit im System, da ein Prinzip der minimalen Rechte eingehalten wird. Demnach kann eine Applikation nur auf zugewiesene und freigegebene Ressourcen im System zugreifen. Ein weiterer Vorteil dieser internen Architektur liegt in der Robustheit des Systems. Wenn eine Applikation durch Fehler terminiert, wird nur der allokierte Prozess beendet und das Betriebssystem bleibt von diesem Problem unberührt. \footcite{Android-SystemPermissions}
Android-Applikationen werden in der Programmiersprache \ac{Java} geschrieben, mit einem Java-\ac{Compiler} kompiliert und dann von einem Cross-Assembler für die entsprechende \ac{VM} aufbereitet. Das Produkt ist ein ausführbares \ac{.apk}-Paket.\footcite[Seite 17-19]{Android-BeckerPant}
Im Folgenden werden die Android-Komponenten, die für die Umsetzung relevant sind, genauer betrachtet.
\subsection{User Interfaces}
\label{ssec:android-ui}
\textit{User Interfaces} sind die Bildschirmseiten der Android-Applikation. Über diese Seiten wird die Benutzerinteraktion geführt. Das \textit{User Interface} besteht aus zwei Arten von Elementen. Zum einen aus \textit{Views}, die es ermöglichen direkte Interaktionen mit dem Benutzer zu führen. Zu nennen sind dabei \textit{Buttons}, Textfelder und Checkboxen. Als zweites werden \textit{View Groups} verwendet, um \textit{Views} sowie andere \textit{View Groups} anzuordnen.
Das \textit{User Interface Layout} ist durch eine hierarchische Struktur gekennzeichnet. Zum Anlegen einer solchen Struktur gibt es verschiedene Möglichkeiten. Zum einen kann man ein \textit{View}-Objekt anlegen und darauf die Elemente platzieren. Aus Gründen der Performance und der Übersicht ist die Möglichkeit einer \ac{XML}-Datei jedoch zielführender. Aus den Knoten der erstellten Datei werden zur Laufzeit \textit{View}-Objekte erzeugt und angezeigt. Die erzeugten \textit{UIs} werden unter \textit{res/layout} im Android-Betriebssystem hinterlegt. Des Weiteren können Ressourcen in den \textit{UIs} verwendet werden. Unter Ressourcen versteht man Elemente, die zum Verzieren von Oberflächen verwendet werden können. Darunter fallen bespielsweise Grafiken oder \textit{Style-Sheets}, die über den jeweiligen Ressourcen-Schlüssel aufgerufen und verwendet werden.\footcite{Android-UI}

\subsection{Activities}
\label{ssec:android-activities}
\textit{Activities} gehören zu den App-Komponenten, da sie ein grundsätzlicher Bestandteil einer Applikation sind. Es gibt im Normalfall mehrere \textit{Activities} in einer App.

Die eigentlichen Aufgaben liegen in der Bereitstellung eines Fensters, das dann auf den Screen, der für die App vom Betriebssystem bereitgestellt wird, gelegt wird. Das Fenster ist im Anschluss für die Annahme von Benutzerinteraktionen bereit. Das Fenster wird mit Hilfe des Aufrufs \textit{SetContentView()} aufgerufen. Zur Benutzerinteraktion werden dann die bereits vorgestellten \textit{View}-Elemente verwendet. Die \textit{Activity} ist folgend für die Verarbeitung und Auswertung der Eingaben verantwortlich.

In jeder Applikation muss es eine \textit{MainActivity} geben, die beim Start der Applikation vom Android-Betriebssystem gestartet wird. Zudem muss eine \textit{Activity} im AndroidManifest mit dem Attribut \textit{Launcher} versehen werden, um diese dann als Einstiegspunkt aus dem Menü des Betriebssystems zu setzen. Dabei ist empfehlenswert, dass dieselbe \textit{Activity} sowohl das Main- als auch Launcher-Attribut erhält.

Diese Festlegungen müssen im Manifest hinterlegt werden. Das Manifest liegt im Root-Ordner der App und stellt dem Betriebssystem wichtige Informationen der Applikation zur Verfügung. Dieses Manifest wird vor Ausführung der App analysiert und ausgewertet. Darin kann beispielweise festgelegt werden, welche Komponenten oder anderen Applikationen auf entsprechende \textit{Activities} zugreifen dürfen. Wenn eine \textit{Activity} nicht von außerhalb der App erreicht werden soll, sollte kein Intent-Filter gesetzt werden, da demnach der genaue Name der \textit{Activity} zum Start bekannt sein muss. Diese Informationen sind jedoch nur in der gegenwärtigen App vorhanden.

Da eine App normalerweise aus mehreren \textit{Activities} besteht, müssen diese \textit{Activities} gestartet werden und untereinander kommunizieren. \textit{Activities} starten sich gegenseitig, weshalb der Aufruf einer \textit{Activity} aus einer anderen erfolgt. Um eine neue \textit{Activity} starten zu können, ist ein Intent von Nöten.
Ein Intent ist ein Nachrichtenobjekt innerhalb von Android, welches zur Kommunikation zwischen App-Komponenten verwendet wird. In diesem Fall zwischen zwei \textit{Activities}. Zur Erstellung benötigt es den Namen der zu startenden Komponente, um eine Verbindung dorthin aufbauen zu können, und eine \textit{Action}, die ausgeführt werden soll. Zudem können Daten übergeben werden, die anschließend als Datenpakete mit dem Aufruf der Komponente mitgegeben werden. Diese Daten sind dann in der gestarteten Komponente aus dem dort vorhandenen Intent auslesbar. Zusätzlich gibt es die Möglichkeit Aktionen vom Betriebssystem ausführen zu lassen. Beispielsweise kann man ein \textit{Intent} mit der Aktion zum Starten des Email-Programms übergeben und die entsprechend im Betriebssystem hinterlegte Applikation zum schreiben von Emails wird geöffnet.

Eine \textit{Activity} kann drei Stati in einem \textit{Lifecycle} einnehmen. Zum einen kann die \textit{Activity} im Status \textit{Resumed} - oft auch \textit{Running} genannt - sein und damit momentan im User-Fokus stehen, also im Vordergrund der App sein und die Interaktionen entgegennehmen. Des Weiteren kann eine \textit{Activity} pausieren, wenn eine andere im User-Fokus steht. Dabei ist der \textit{View} der betrachteten \textit{Activity} jedoch immer noch teilweise sichtbar, da der darüberliegende \textit{View} zu Beispiel nicht den gesamten Bildschirm in Anspruch nimmt. Anders verhält es sich, wenn der \textit{View} der betrachteten \textit{Activity} komplett überdeckt ist. Dann befindet sich die \textit{Activity} nämlich im Status \textit{Stopped}. Sowohl im Status \textit{Stopped} als auch im Status \textit{Paused} lebt die \textit{Activity} noch. Das bedeutet, dass das \textit{Activity}-Objekt zusammen mit allen Objekt-Stati und Memberinformationen im Arbeitsspeicher liegt. Der einzige Unterschied dieser beiden Stati liegt darin, dass eine \textit{Activity} im Status \textit{Paused} noch eine Verbindung zum \textit{WindowManager} besitzt, die im Status \textit{Stopped} nicht mehr vorhanden ist. Gemeinsam haben diese beiden Stati jedoch noch, dass sie bei mangelndem Arbeitsspeicher vom Betriebssystem zerstört werden können. 

%ActivityLifecycle-Bild
\begin{figure}[h]
\centering
\includegraphics[width=0.8\linewidth]{content/images/Android-ActivityLifecycle}
\caption{Android Activity-Lifecycle}
Quelle: https://developer.android.com/guide/components/activities.html
\label{pic:androidActivityLifecycle}
\end{figure}

Die Ausführung der internen Methoden einer \textit{Activity} ist abhängig von den Eingaben des Benutzers. Dabei durchläuft jede \textit{Activity} ihren \textit{Lifecycle}, der in Abbildung~\ref{pic:androidActivityLifecycle} dargestellt ist. Darin ist zu erkennen, dass zuerst die \textit{OnCreate()}-Methode aufgerufen wird. Darin werden alle essentiellen Initialisierungen gemacht und der \textit{View} aufgerufen. Nachfolgend werden \textit{OnStart()} und \textit{OnResume()} durchlaufen bis die \textit{Activity} den User-Fokus wieder verliert, jedoch der \textit{View} noch sichtbar ist. In dem Moment wird die \textit{OnPause()}-Methode ausgeführt, um Benutzereingaben gegebenenfalls speichern zu können, denn in diesem Zustand ist es in seltenen Fällen möglich, dass der Status - wie oben erklärt - durch das Betriebssystem zerstört wird. Kehrt der Benutzer zurück, wird \textit{OnResume()} wieder aufgerufen, sonst \textit{OnStop()}, um auch dort aufgenommene Daten persitieren zu können. Von dort gibt es zwei verschiedene Rücksprung-Möglichkeiten. Zum einen könnte der Fall eintreten, dass die Daten der \textit{Activity} aus dem Arbeitsspeicher gelöscht wurden, die \textit{Activity} jedoch noch einmal aufgerufen wird. In diesem Fall startet die \textit{Activity} wieder von vorn. Eine weitere Möglichkeit ist die Rückkehr des Benutzers zu der \textit{Activity}. Dabei werden dann die Methoden \textit{OnRestart()} und \textit{OnStart()} aufgerufen.

Zusammenfassend lässt sich daraus ableiten, dass die Persistierung von Eingaben in den Methoden \textit{OnPause()}, \textit{OnStop()} und \textit{OnDestroy()} durchgeführt werden sollten, da diese Zustände zerstört werden können. Die weiteren Methoden sollten aus Performancegründen jedoch minimal und agil gehalten werden.

\subsection{Services}
\label{ssec:android-services}
\textit{Services} sind, genauso wie \textit{Activities}, App-Komponenten, die zu den Grundbausteinen einer Android-App gehören. \textit{Services} unterscheiden sich jedoch hinsichtlich ihrer Aufgaben stark von \textit{Activities}. So sind sie dazu da, Aufgaben im Hintergrund zu erledigen. Zudem besitzen sie keinen zugehörigen \textit{View}, sondern werden von anderen App-Komponenten, wie beispielsweise einer \textit{Activity} gestartet. Sie laufen im \textit{Main-Thread} des Prozesses der aufrufenden Komponente. Ein \textit{Service} erstellt keinen eigenen \textit{Thread}, noch einen eigenen Prozess zur Abarbeitung der Aufgaben. Diese Eigenschaft der \textit{Services} muss vom Entwickler bedacht werden. Denn daraus kann man ableiten, dass rechenintensive Aufgaben in einem explizit gestarteten \textit{Thread} arbeiten sollten, um Fehler der Art \textit{Application Not Rsponding} (ANR) zu vermeiden und die Benutzeroberfläche nicht unnötig zu verlangsamen. Ein Vorteil besteht jedoch darin, dass \textit{Services} Aufgaben auch dann noch ausführen können, wenn die App, zu der sie gehören, geschlossen wurde. So können noch nicht abgeschlossene \textit{Up-} oder \textit{Downloads} noch beendet werden oder das Abspielen von Musik bei ausgeschaltetem Bildschirm fortgeführt werden.

Bei Android wird grundsätzlich zwischen zewi Arten von \textit{Services} unterschieden. Zum einen gibt es \textit{Started-Services}, die durch eine App-Komponente mit dem Befehl \textit{StartService()} gestartet werden. Grundsätzlich ist dieser Aufruf uneingeschränkt von allen App-Komponenten möglich, soweit die Einstellungen im Android-Manifest diese zulassen. Weiterhin laufen \textit{Started-Services} im Hintergrund der App weiter, auch wenn die Komponente, die den \textit{Service} gestartet hat, zerstört oder beendet wurde. Deshalb führt diese Art des \textit{Services} im Normalfall eine Aufgabe aus und stoppt sich anschließend nach der Fertigstellung selbstständig. Auf der anderen Seite gibt es \textit{Bound-Services}, die durch einen Aufruf von \textit{BindServcice()} einer anderen App-Komponente gestartet werden. In diesem Schritt verbinden sich die Komponente und der \textit{Service} über eine Art \textit{Client-Server Interface}, das zur Kommunikation bereitgestellt wird. Dieses \textit{Interface} ist vom Typ \textit{IBind} und sorgt für den Austausch von \textit{Request} und \textit{Results}. Des Weiteren verläuft eine mögliche Interprozess-Kommunikation zwischen Komponente und \textit{Service} über dieses \textit{Interface}. Die größte Besonderheit eines \textit{Bound-Services} besteht darin, dass der \textit{Service} nur so lange besteht, wie mindestens eine Komponente an diesen gebunden ist. Natürlich ist es möglich, dass sich mehrere Komponenten gleichzeitig an diesen \textit{Service} binden können. Löst sich jedoch die letzte Komponente wieder, wird der \textit{Service} zerstört. Natürlich gibt es Mischformen dieser beiden \textit{Service}-Arten, die abhängig von der zu leistenden Aufgabe gewählt werden sollten. 

%ServiceLifecycle-Bild
\begin{figure}[h]
\centering
\includegraphics[width=0.8\linewidth]{content/images/Android-ServiceLifecycle}
\caption{Android Service-Lifecycle}
Quelle: https://developer.android.com/guide/components/services.html
\label{pic:androidServiceLifecycle}
\end{figure}

Zum Erstellen eines \textit{Services} muss von der Klasse \textit{Service}, oder davon abgeleitete Klassen, geerbt werden. Danach müssen die vorgegebenen Methoden überschrieben werden, denn \textit{Services} besitzen, genauso wie \textit{Activities}, einen Lebenszyklus. Dabei muss jedoch wieder zwischen den beiden Arten von \textit{Services} unterschieden werden.

\textit{Started-Services} werden die Methode \textit{OnCreate()} nach dem Start durch eine Komponente ausführen, wenn der \textit{Service} noch nicht läuft. Darin sollten dann die Initialisierungen und einmaligen Aufgaben zum Start des \textit{Services} durchgeführt werden. \textit{OnStartCommand()} wird immer dann aufgerufen, wenn der \textit{Service} wieder von einer Komponente aufgerufen wird. Dann befindet er sich im Zustand \textit{Running} und führt die ihm zugewiesenen Aufgaben durch. Wenn der \textit{Service} zerstört wird, sei es durch Speichermangel des \textit{Devices} oder das Beenden durch eine Komponente oder den \textit{Service} selbst, wird \textit{OnDestroy()} ausgeführt, um abschließende Aufgaben durchzuführen. Dazu zählen beispielsweise das Beenden von Datenbankverbindungen oder \textit{Threads}.

\textit{Bound-Services} werden, wie oben genannt, über \textit{BindService()} von einer Komponente gestartet und führen dann, genauso wie die \textit{Started-Services}, die \textit{OnCreate()}-Methode zum Initialisieren aus. Gefolgt vom aktiven Status, in dem anfangs \textit{OnBind()} aufgerufen wird und die von den Komponenten verlangten Aufgaben ausgeführt werden. Anschließend lösen sich die Komponenten wieder vom \textit{Service}. Haben sich alle Komponenten gelöst, wird auch beim \textit{Bound-Service} \textit{OnDestroy()} ausgeführt.


\subsection{Threading/Asynchronität}
\label{ssec:android-threading-async}

\subsection{SQLite}
\label{ssec:android-sqlite}

\section{Was ist XAMARIN?}
\label{sec:defintion-xamarin}

\subsection{Multiplattform-Unterstützung}
\label{ssec:xamarin-multiplattform}

\subsection{Besonderheiten der Android-Umsetzung}
\label{ssec:xamarin-android}
Man muss z.B. Activities nicht manuell im Manifest eintragen -> macht XAMARIN für einen!

\section{Eigene Umsetzung}
\label{sec:nat-umsetzung}

\subsection{Anlegen der Layouts}
\label{ssec:nat-layouts}

\subsection{Konnektivität zum Server}
\label{ssec:nat-konnektivität}

\subsection{Lokale Datenbank}
\label{ssec:nat-db}

\subsection{Umsetzung des Caches}
\label{ssec:nat-cache}

	\chapter{Realisierung der clientseitigen Implementierung als Webapplikation}
\label{cha:web-app}

	\chapter{Gegenüberstellung der clientseitigen Implementierungen}
\label{cha:gegenueberstellung}
Ziel dieses Kapitels ist die Gegenüberstellung der Erkenntnisse zur Entwicklung einer verlässlichen mobilen Applikation. Hierbei wird das neu erlangte Wissen zur Umsetzung einer Applikation als \ac{SPA} und als native App bewertet, so dass mit der vorteilhafteren der beiden Optionen der Messeprototyp umgesetzt werden kann. \\
Das Kapitel schließt auch gleichzeitig die Entwicklung des Meilensteins 1 ab.

\section{Umsetzung als SPA}
\label{sec:gegenueberstellung-SPA}
Als Erstes sollen die Vor- und Nachteile der Umsetzung des Clients als Single Page Application aufgezeigt werden.

\subsection{Vorteile}
\label{sec:vorteile-SPA}
Bei der Umsetzung des Clients als Web Applikation zeigen sich die Vorteile besonders in der Umsetzung der Oberfläche. \\
Durch die Nutzung aktueller Web-Techniken und unter Nutzung geeigneter Frameworks lässt sich sehr leicht ein einheitliches Aussehen schaffen, welche für verschiedene Anzeigegrößen optimiert wurde. Hierbei ist man nicht nur auf mobile Endgeräte beschränkt sondern erhält quasi nebenbei eine Webseite, die bequem eine Desktop-Anwendung ersetzen kann. Auch die Umsetzung der Business-Logik konnte ohne großen Einarbeitung-Aufwand bewerkstelligt werden. Dabei fällt auf, dass durch das Voranschreiten von HTML5 viele Funktionen, welche vor einigen Jahren nur durch Desktop Applikationen umgesetzt werden, heute schon problemlos im Browser abbildbar sind. Hierbei zeigten sich aber auch die Schwächen einer Umsetzung als Web Applikation. 

\subsection{Nachteile}
\label{sec:nachteile-SPA}
Wie bereits erwähnt sind viele, aber noch nicht alle Techniken für den Browser umgesetzt. So ist die Umsetzung der \textit{IndexedDB} für iOS und Microsoft-Geräte noch sehr fehleranfällig\footcite{online:caniuse:indexedDB}. In diesem Punkt spiegelt sich auch das größte Problem jeder Web-Umsetzung wieder: Unterschiedliche Browser implementieren einige APIs anders oder teilweise auch gar nicht, sodass vieles der Entwicklungszeit für das Anpassen der Funktionen und Oberflächen für die verschiedenen Browser genutzt werden muss. Wenn es nun so ist, dass Kern-Komponenten wie in unserem Fall die IndexedDB in einigen wichtigen Browsern (iOS Safari-Nutzung bei 7.33\% (v. 8.1-8.4 Stand 31.08.2015\footcite{online:caniuse:indexedDB}) nicht ausreichen unterstützt werden, ist die Umsetzung dieses Teilaspekts für den produktiven Einsatz fast unmöglich. \\
Ein weiterer Nachteil ergibt sich aus der Nutzung von AngularJs. Da die gesamte Datenaufbereitung mit Authentifizierung und dem Routing auf Seiten des Clients passiert, können die lokal gespeicherten Daten mit Hilfe der Entwicklungswerkzeuge des Browsers einfach ausgelesen werden. Darum wäre es unter Sicherheitsaspekten fahrlässig, die hier vorgestellte Implementierung der Authentifizierung (siehe Kapitel \ref{ssec:statusloses-http}) ohne weitere Sicherheitsmaßnahmen produktiv zu stellen.

\section{Umsetzung als native App}
\label{sec:gegenueberstellung-native-app}
Des Weiteren sollen auch die Vor- und Nachteile der Umsetzung als native Applikation vorgestellt werden.

\subsection{Vorteile}
\label{sec:vorteile-native-app}
Die Vorteile einer nativen Applikation liegen besonders in dem umfangreichen Funktionsumfang. Dieser kann alle bereitgestellten Funktionen des Betriebssystems ausnutzen. Dazu zählen das \textit{Threading} (siehe Kapitel \ref{ssec:android-prozesse-threads}) und interne Aufrufe über \textit{Services}, die dann zum Beispiel zum Versenden vom Emails verwendet werden können. Zudem können Daten persistent, auch über die Dauer einer \textit{Session} hinaus, auf dem Gerät gespeichert werden (siehe Lokale Datenbank in Kapitel \ref{ssec:nat-db}).\\
Eine hohe Sicherheit kann in dem Zuge einer nativen App ebenfalls bereitgestellt werden, da die Daten nur mit guten technischen Kenntnissen ausgelesen werden können. Sind die Daten darüber hinaus noch lokal verschlüsselt, so sind diese sicher.\\
Weiterhin ist die gesamte Logik der Applikation nicht sichtbar für den Endanwender und von Manipulationen bei Datenabrufen kann man ausschließen. Diese sind nur über aufwendige programmatische Eingriffe möglich.

\subsection{Nachteile}
\label{sec:nachteile-native-app}
Die Umsetzung der nativen App beansprucht viel Zeit und ein grundlegendes Knowhow über die Funktionsweise des zu unterstützende Betriebssystems.\\
Darin liegt auch noch ein weiteres Problem. Um eine große Markt-Abdeckung mit einer nativen App zu erreichen, benötigt man mindestens eine iOS- und eine Android-Applikation. Dann besitzt den Zugang zu über 95\% der Smartphone-Nutzer in Deutschland.\footcite{Statista-SmartphoneVerteilung}\\
Die Entwicklung für zwei Systeme kann daraufhin in zwei Möglichkeiten umgesetzt werden. Zum einen könnten native Apps in den jeweiligen Sprachen entwickelt werden. Zum anderen kann eine Multiplattform-Lösung, wie Xamarin-Platform es ist, eingesetzt werden. Dabei beschränkt sich der Funktionsumfang dann aber auf die grundlegenden Funktionen, wenn bei den verschiedenen Funktionen nicht noch zwischen den Systemen unterschieden wird.\\
Weiterhin ist das Erstellen von Oberflächen aufwendiger als bei einer \textit{Single Page Application}.\\
Zudem müssen native Apps direkt auf das \textit{Device} geladen werden und können nicht einfach über das Internet aufgerufen werden.

\section{Fazit aus Meilenstein 1}
\label{sec:gegenueberstellung-fazit}
Zusammenfassend kann festgehalten werden, dass die Nachteile der \textit{Single Page Application} dahingehend überwiegen, dass die Sicherheit der Daten - besonders in Verbindung mit Vitaldaten - eine höhere Priorität einnimmt. Diese Anforderung kann nur von einer nativen App zufriedenstellend geleistet werden. Hierbei zeigt sich die bisher unzureichende Implementierung der Datenbank in einigen Browsern zum Zeitpunkt der Projektdurchführung als unzureichend für die Anforderungen einer Anwendung, welche auf verschiedenen gängigen Mobilgeräten laufen soll. Ändert sich der Zustand der Implementierung müsste diese Evaluation neu durchgeführt werden. 

	\chapter{Fazit}
\label{fazit}
In diesem Kapitel wird eine Retrospektive des durchgeführten Projekts gegeben. Dabei wird zwischen dem Ergebnis bei der Erstellung der Applikationen und dem Erreichen der persönlichen Ziele unterschieden.
\section{Ziele / Ergebnisse}
\label{sec:ziele-ergebnisse}
Rückblickend kann das Projekt als ein Erfolg gesehen werden. \\
Aus dem Projekt ist eine umfassende Client-Server-Landschaft entstanden, welche die gesetzten Anforderungen an eine verlässliche mobile Applikation, sowohl in den Aspekten der Benutzerfreundlichkeit als auch der Verlässlichkeit, erfüllt. Sowohl der Server als auch beide Clients haben in den jeweils erstellten Meilensteine alle muss Kriterien erfüllt. Teilweise konnten sogar noch optionale Kann-Kriterien umgesetzt werden. So ist es beispielsweise möglich, sich an der Web \ac{API} zu registrieren oder in der nativen \gls{App} eine Trainingsstatistik aufzubauen. 
Dies ist besonders deshalb bemerkenswert, da die Projektteilnehmenden einen großteils der genutzten Techniken erst erlernen mussten. 
Alle im Projekt erzeugten Ressourcen können über das öffentliche Git-Repository unter \textit{\href{https://github.com/Fanuer/fIT}{https://github.com/Fanuer/fIT}} abgerufen werden. 

%Anfangs wurde ein Überblick über die Aufgabe gegeben und die Systemarchitektur vorgestellt. Darauf aufbauend wurden die verwendeten Technologien erörtert und die Umsetzungen der beiden Applikationen bis zu einem festgelegten Punkt dokumentiert. Die Implementierung des Servers wurde zusätzlich betrachtet. Die anschließende Gegenüberstellung der Apps hat ergeben, dass die Weiterentwicklung der Android-App in Hinsicht auf Sicherheit als einzige Lösung gesehen werden kann.\\
%Deshalb wurde für diese Applikation ein zweiter Meilenstein begonnen, um die Anforderungen an den umzusetzenden Prototypen zu implementieren. 
%Rückblickend kann das Projekt als ein großer Erfolg gesehen werden. Die eingangs formulierten Ziele konnten vollständig umgesetzt werden. Zudem wurden teilweise noch Funktionen umgesetzt, die über das formulierte Ziel hinaus gehen.\\
%Zusammenfassend ist es nun mit dem Prototypen möglich den kompletten Funktionsumfang der nativen Android-App auch im \textit{Oflline}-Modus nutzen zu können. 
\section{Erkenntnisse}
\label{sec:erkenntnisse}
Der Erkenntnisgewinn dieser Arbeit ist beträchtlich. \\
Es wurden vorwiegend unbekannte und für die Autoren neue Technologien verwendet. Die Einarbeitung geschah in den meisten Fällen reibungslos. Dies wurde gerade zu Anfang des Projekts jedoch auch als Risiko empfunden, welches ein Problem in der Umsetzung hätte verursachen können. Diese Zweifel waren jedoch im nach hinein unbegründet. \\
Die Tatsache, dass dieses Projekt von zwei Personen durchgeführt wurde, ergab sich als enormer Vorteil. Dies gestattet, punktuell Spezialwissen zu erwerben, was jeweils der anderen Person vermittelt werden musste. Dadurch wurde der Lerneffekt nochmals verstärkt und Zusammenhänge leichter vertieft.\\
Ebenfalls wurde von beiden Projektteilnehmer als sinnvoll erachtet, dass eine Komplettlösung für ein größeres Feld an Aufgaben, nämlich mobile Applikationen, geschaffen werden musste. Das meint, dass nicht nur einer Cache-Komponente für ein bestehendes System erstellt wurde, sondern jede Komponenten für die Realisierung der Systemlandschaft erstellt, verwaltet und veröffentlicht werden musste.\\
Somit konnte sich ein Einblick über alle anfallenden Arbeitsschritte zur Umsetzung einer mobilen Applikation geschaffen werden. 
\section{Ausblick}
\label{sec:ausblick}
Die Grundlage für den Ausbau dieses Projektes zu einer marktreifen \gls{App} ist gegeben. Der dahinter liegende Server ist stark genug, um eine größere Last an Anfragen zu bewältigen. Zum Ausbau dieses Projekts muss die \gls{Android}-\gls{App} weiterentwickelt werden. \\
Die Umsetzung aller wichtiger Funktionen wurde jeweils an mindestens einem Beispiel im Messeprototyp dargestellt und muss demnach nur noch auf die fehlenden Funktionalitäten übertragen werden.\\
Durch den nun tieferen Einblick in die Technologien sollte sich dieser Aufwand in Grenzen halten.\\
Auf Seiten dies Servers müssen hauptsächlich Sicherheitsfunktionen weiterentwickelt werden, um die Web API öffentlich nutzbar zumachen. Zu diesen Funktionen zählen beispielsweise die Validierung von ClientIDs, die Einschränkung eingehender Anfragen durch \ac{CORS} und die Signierung von Tokens. \\
Darüber hinaus kann eine Weiterentwicklung der \ac{App} dazu führen, dass die Web \ac{API} Ressourcen erweitert werden muss. 
	
	% ----------------- Ende des eigentlichen Textes
	

	%Verzeichnisse erstellen
  \chapter*{Abkürzungsverzeichnis}
\begin{acronym}[BiPRO ] %Längster Begriff
\setlength{\itemsep}{-\parsep}
	\acro{ACL}{Access Control Lists}
	\acro{AES}{Advanced Encryption Standard}
	\acro{ANR}{Application Not Responding}
	\acro{API}{Application Programming Interface}
	\acro{CORS}{Cross-origin resource sharing}
	\acro{CRUD}{Create Read Update Delete}
	\acro{DRAM}{Dynamischer RAM}
	\acro{HTTP}{Hyper Text Transfer Protocol}
	\acro{OR-Mapper}{objekt-relationaler Mapper}
	\acro{PCL}{Portable Class Library}
	\acro{RAM}{Random Access Memory}
	\acro{SRAM}{Statischer RAM}
	\acro{TCP}{Transmission Control Protocol}
	\acro{URI}{Uniform Resource Identifier}
	\acro{VM}{Virtuelle Maschine}
	\acro{XML}{Extensible Markup Language}
	\acro{.apk}{Android Package}
	%usw.
\end{acronym}

	\listoffigures
	\listoftables
	\lstlistoflistings
	
	\appendix
	\nocite{*} 
	
	%Literaturverzeichnis erstellen
	\bibliography{bib/bib}

	%\chapter{Anhang}
\label{cha:anhang}

\includepdf[pages={-},frame= true, scale=0.68, pagecommand=\section{Pflichtenheft}\label{sec:Pflichtenheft}]{content/additional/Pflichtenheft.pdf}

\section{Cache Post}

%CachePush-Bild
\begin{figure}[h]
\centering
\includegraphics[width=0.8\linewidth]{content/images/Cache-Post}
\caption{Hochladen zum Server}
\label{pic:cachePost}
\end{figure}
	\addcontentsline{toc}{chapter}{Eidesstattliche Erkl{\"a}rung}  
\chapter*{Eidesstattliche Erklärung}
\label{cha:eid-erklaerung}
Gemäß §\,17,(5) der BPO erkläre ich an Eides statt, dass ich die vorliegende Arbeit
selbständig angefertigt habe. Ich habe mich keiner fremden Hilfe bedient und keine
anderen, als die angegebenen Quellen und Hilfsmittel benutzt. Alle Stellen, die
wörtlich oder sinngemäß veröffentlichten oder nicht veröffentlichten Schriften und
anderen Quellen entnommen sind, habe ich als solche kenntlich gemacht. Diese
Arbeit hat in gleicher oder ähnlicher Form noch keiner Prüfungsbehörde vorgelegen.
\vspace{3\baselineskip}\\
Dortmund, \thedate \hfill \theauthor

\vspace{1cm}
\section*{Erklärung}
\label{sec:erklaerung}
Mir ist bekannt, dass nach §\,156~StGB bzw. §\,163~StGB eine falsche Versicherung
an Eides Statt bzw. eine fahrlässige falsche Versicherung an Eides Statt mit
Freiheitsstrafe bis zu drei Jahren bzw. bis zu einem Jahr oder mit Geldstrafe
bestraft werden kann.
\vspace{3\baselineskip}\\
Dortmund, \thedate \hfill \theauthor

\end{document}
